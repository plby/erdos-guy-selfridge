%\pdfoutput=1
\documentclass[12pt,a4paper,reqno]{amsart}
\newcommand\hmmax{0}
\newcommand\bmmax{0}
\usepackage{amssymb}
\usepackage{amscd}
\usepackage[pdftex,pdfpagelabels]{hyperref}
\usepackage{enumerate}
\usepackage{comment}
%\usepackage{psfig}
\usepackage{graphicx}
\usepackage{cleveref}
\usepackage{siunitx}
\usepackage{tikz-cd}
\usepackage{stix}
\usepackage{bm}
\DeclareMathAlphabet\mathbfcal{LS2}{stixcal}{b}{n}
\numberwithin{equation}{section}

%\usepackage{mathabx}


\usepackage{mathtools}%                  http://www.ctan.org/pkg/mathtools
\usepackage[tableposition=top]{caption}% http://www.ctan.org/pkg/caption
\usepackage{booktabs,dcolumn}%           http://www.ctan.org/pkg/dcolumn + http://www.ctan.org/pkg/booktabs

% Lighter notation.
%\newcommand*\mc[1]{\multicolumn{1}{c}{#1}}
%\newcommand*\tupref[2]{\href{http://math.mit.edu/~primegaps/tuples/admissible_#1_#2.txt}{\num{#2}}}



%\DeclareMathOperator*\Kl{Kl} (commented because yield bad display for  \Kl_q %replaced with \newcommand... )
%\DeclareMathOperator*\FT{FT} (commented because yield bad display for \FT_q
%replaced with \newcommand...)

\DeclareMathOperator*\swan{swan}
\DeclareMathOperator*\cond{cond}
\DeclareMathOperator*\Gal{Gal}

\newcommand{\FT}{\mathrm{FT}}
\newcommand{\Kl}{\mathcal{K}\ell}
% Setup for ``caption''.
%\DeclareCaptionLabelSeparator{separation}{:\quad}
%\captionsetup{
  %font=small,
  %labelfont=sc,
  %labelsep=separation,
  %width=0.8\textwidth
%}

\DeclareFontFamily{OT1}{rsfs}{}
\DeclareFontShape{OT1}{rsfs}{n}{it}{<-> rsfs10}{}
\DeclareMathAlphabet{\mathscr}{OT1}{rsfs}{n}{it}

\addtolength{\textwidth}{3 truecm}
\addtolength{\textheight}{1 truecm}
\setlength{\voffset}{-.6 truecm}
\setlength{\hoffset}{-1.3 truecm}

\theoremstyle{plain}

\newtheorem{theorem}{Theorem}[section]
%\newtheorem{theorem}[theorem]{Theorem}
\newtheorem{proposition}[theorem]{Proposition}
\newtheorem{lemma}[theorem]{Lemma}
\newtheorem{corollary}[theorem]{Corollary}
\newtheorem{conjecture}[theorem]{Conjecture}
\newtheorem{heuristic}[theorem]{Heuristic}
\newtheorem{principle}[theorem]{Principle}
\newtheorem{question}[theorem]{Question}
\newtheorem{problem}[theorem]{Problem}
\newtheorem{claim}[theorem]{Claim}

\theoremstyle{definition}

%\newtheorem{roughdef}[subsection]{Rough Definition}
\newtheorem{definition}[theorem]{Definition}
\newtheorem{remark}[theorem]{Remark}
\newtheorem{remarks}[theorem]{Remarks}
\newtheorem{example}[theorem]{Example}
\newtheorem{examples}[theorem]{Examples}
%\newtheorem{problem}[subsection]{Problem}
%\newtheorem{question}[subsection]{Question}

\renewcommand\P{\mathbb{P}}
\newcommand\E{\mathbb{E}}
\newcommand\Var{\mathrm{Var}}
\newcommand\R{\mathbb{R}}
\newcommand\Z{\mathbb{Z}}
\newcommand\F{\mathbf{F}}
\newcommand\N{\mathbb{N}}
\newcommand\n{\mathbf{n}}
\renewcommand\a{\mathbf{a}}
\renewcommand\b{\mathbf{b}}
\renewcommand\j{\mathbf{j}}
\renewcommand\k{\mathbf{k}}
\renewcommand\v{\mathbf{v}}
\renewcommand\t{\mathbf{t}}
\renewcommand\r{\mathbf{r}}
\renewcommand\l{\mathbf{l}}
\newcommand\X{\mathbf{X}}
\newcommand\T{\mathbf{T}}
\newcommand\Y{\mathbf{Y}}
\newcommand\A{\mathbf{A}}
\newcommand\W{\mathbf{W}}
\newcommand\C{\mathbb{C}}
\newcommand\Q{\mathbb{Q}}
\renewcommand\Re{{\operatorname{Re}}}
\renewcommand\Im{{\operatorname{Im}}}
\newcommand\Log{{\operatorname{Log}}}
\newcommand\lcm{{\operatorname{lcm}}}
\renewcommand\gcd{{\operatorname{gcd}}}
\newcommand\eps{\varepsilon}
\newcommand\tuple{{\mathcal B}}
\newcommand\excess{{\mathcal{E}}}

\renewcommand{\mod}{\bmod}

\parindent 0mm
\parskip   5mm


\begin{document}

\title{Decomposing a factorial into large factors}

\author{Boris Alexeev}
\address{???}
\email{???}


\author{...}
\address{???}
\email{???}

\author{Terence Tao}
\address{UCLA Department of Mathematics, Los Angeles, CA 90095-1555.}
\email{tao@math.ucla.edu}

\author{Markus Uhr}
\address{???}
\email{???}

\author{Kevin Ventullo}
\address{Google, Mountain View, CA}
\email{kevinventullo@google.com}


\subjclass[2020]{11A51}

\begin{abstract}  Let $t(N)$ denote the largest number such that $N!$ can be expressed as the product of $N$ numbers greater than or equal to $t(N)$.
The bound $t(N)/N = 1/e-o(1)$ was apparently established in unpublished work of Erd\H{o}s, Selfridge, and Straus; but the proof is lost.  Here we obtain the more precise asymptotic 
$$ \frac{t(N)}{N} = \frac{1}{e} - \frac{c_0}{\log N} + O\left( \frac{1}{\log^{1+c} N} \right)$$
for an explicit constant $c_0 = 0.3044190\dots$ and some absolute constant $c>0$, answering a question of Erd\H{o}s and Graham.  With numerical assistance, we also establish several conjectures of Guy and Selfridge concerning effective estimates of this quantity, for instance establishing $t(N) \geq N/3$ for $N \geq 43632$, with the threshold shown to be best possible. 
\end{abstract}


\maketitle

%%%%%%%%%%%%%%%%%%%%%%%%%%%%%%%%%%%%%%%%%%%%%%%%%

\section{Introduction}

Given a natural number $M$, define a \emph{factorization} of $M$ to be a finite multiset $\tuple$ such that the product
$$ \prod \tuple \coloneqq \prod_{a \in \tuple} a$$
(where the elements are counted with multiplicity) is equal to $M$; more generally, define a \emph{subfactorization} of $M$ to be a finite multiset $\tuple$ such that $\prod \tuple$ divides $M$.  Given a threshold $t$, we say that a multiset $\tuple$ is \emph{$t$-admissible} if $a \geq t$ for all $a \in \tuple$.  For a given natural number $N$, we then define $t(N)$ to be the largest $t$ for which there exists a $t$-admissible factorization $\tuple$ of $N!$ of cardinality $|\tuple|=N$.  

\begin{example}\label{nine}  The multiset
  $$ \{ 3,3,3,3,4,4,5,7,8\}$$
  is a $3$-admissible factorization of 
$$ \prod \{ 3,3,3,3,4,4,5,7,8\} = 3^4 \times 4^2 \times 5 \times 7 \times 8 = 9!$$
of cardinality 
$$|\{ 3,3,3,3,4,4,5,7,8\}| = 9,$$
 hence $t(9) \geq 3$.  One can check that no $4$-admissible factorization of $9!$ of this cardinality exists, hence $t(9) = 3$.
\end{example}

It is easy to see that $t(N)$ is non-decreasing in $N$, (any cardinality $N$ factorization of $N!$ can be extended to a cardinality $N+1$ factorization of $(N+1)!$ by adding $N+1$ to the multiset).  The first few elements of the sequence $t(N)$ are
$$ 1,1,1,2,2,2,2,2,3,3,3,3,3,4, \dots$$
(\href{https://oeis.org/A034258}{OEIS A034258}). The values of $t(N)$ for $N \leq 79$ were computed in \cite{guy}, and the values for $N \leq 200$ can be extracted from \href{https://oeis.org/A034259}{OEIS A034259}, which describes the inverse sequence to $t$.  As part of our work, we extend this sequence to $N \leq 10^4$; see \cite{github}, \Cref{fig-long}.

When the factorial $N!$ is replaced with an arbitrary number, this problem is essentially the bin covering problem, which is known to be NP-hard; see e.g., \cite{bincover}.  However, as we shall see in this paper, the special structure of the factorial (and in particular, the profusion of factors at the ``tiny primes'' $2,3$) make it more tractable to estimate $t(N)$ with high precision than in the general case.

\begin{remark}\label{subfac}  One can equivalently define $t(N)$ as the greatest $t$ for which there exists a $t$-admissible \emph{subfactorization} of $N!$ of cardinality \emph{at least} $N$.  This is because every such subfactorization can be converted into a $t$-admissible factorization of cardinality exactly $N$ by first deleting elements from the subfactorization to make the cardinality $N$, and then multiplying one of the elements of the subfactorization by a natural number to upgrade the subfactorization to a factorization.  This ``relaxed'' formulation of the problem turns out to be more convenient for both theoretical analysis of $t(N)$ and numerical computations.
\end{remark}

By combining the obvious lower bound
\begin{equation}\label{obvious}
 \prod \tuple \geq t^{|\tuple|}
\end{equation}
for any $t$-admissible multiset $\tuple$ with Stirling's formula \eqref{stirling}, we obtain the trivial upper bound
\begin{equation}\label{trivial} \frac{t(N)}{N} \leq \frac{(N!)^{1/N}}{N} = \frac{1}{e} + O\left( \frac{\log N}{N}\right)
\end{equation}
for $N \geq 2$; see \Cref{fig1}.  In \cite[p.75]{erdos-graham} it was reported that an unpublished work of Erd\H{o}s, Selfridge, and Straus established the asymptotic
\begin{equation}\label{t1}
  \frac{t(N)}{N} = \frac{1}{e} + o(1)
\end{equation}
(first conjectured in \cite{erdos-71}) and asked if one could show the bound
\begin{equation}\label{Tbound}
   \frac{t(N)}{N} \leq \frac{1}{e} - \frac{c}{\log N}
\end{equation}
for some constant $c>0$ (problem {\#}391 in \url{https://www.erdosproblems.com}; see also \cite[Section B22, p. 122--123]{guy}); it was also noted that similar results were obtained in \cite{algr77} if one restricted the $a_i$ to be prime powers.  However, as later reported in \cite{erdos-96}, Erd\H{o}s ``believed that Straus had written up our proof [of \eqref{t1}]. Unfortunately Straus suddenly died and no trace was ever found of his notes. Furthermore, we never could reconstruct our proof, so our assertion now can be called only a conjecture''.   In \cite{guy} the lower bound $\frac{t(N)}{N} \geq \frac{1}{4}$ was established for sufficiently large $N$, by rearranging powers of $2$ and $3$ in the obvious factorization $1 \times 2 \times \dots \times N$ of $N!$.  A variant lower bound of the asymptotic shape $\frac{t(N)}{N} \geq \frac{3}{16}-o(1)$ obtained by rearranging only powers of $2$, and which is superior for medium values of $N$, can also be found in \cite{guy}.  The following conjectures in \cite{guy} were also made:
\begin{enumerate}
\item One has $t(N) \leq N/e$ for $N \neq 1,2,4$.
\item One has $t(N) \geq \lfloor 2N/7 \rfloor$ for $N \neq 56$.
\item One has $t(N) \geq N/3$ for $N \geq 3 \times 10^5$.  (It was also asked if the threshold $3 \times 10^5$ could be lowered.) 
\end{enumerate}

\begin{figure}
  \centering
  \includegraphics[width=0.8\textwidth]{newplot_200.png}
  \caption{The function $t(N)/N$ (blue) for $N \leq 200$, using the data from \href{https://oeis.org/A034258}{OEIS A034258}, as well as the trivial upper bound $(N!)^{1/N}/N$ (green), the improved upper bound from \Cref{upper-crit} (pink), which is asymptotic to \eqref{asym} (purple), and the function $\lfloor 2N/7 \rfloor/N$ (brown), which we show to be a lower bound for $N \neq 56$.  \Cref{main} implies that $t(N)/N$ is asymptotic to \eqref{asym} (purple), which in turn converges to $1/e$ (orange).  The threshold $1/3$ (red) is permanently crossed at $N=43632$. 
  }\label{fig1}
  \end{figure}
  
  \begin{figure}
    \centering
    \includegraphics[width=0.8\textwidth]{newplot_600_all.png}
    \caption{A continuation of \Cref{fig1} to the region $80 \leq N \leq 599$. }\label{fig1-alt}
  \end{figure}
  
In this paper we answer all of these questions.

\begin{theorem}[Main theorem]\label{main} Let $N$ be a natural number.
\begin{itemize}
\item[(i)] If $N \neq 1,2,4$, then $t(N) \leq N/e$.
\item[(ii)]  If $N \neq 56$, then $t(N) \geq \lfloor 2N/7 \rfloor$.
\item[(iii)]  If $N \geq 43632$, then $t(N) \geq N/3$.  The threshold $43632$ is best possible.
\item[(iv)]  For large $N$, one has
  \begin{equation}\label{asym}
    \frac{t(N)}{N} = \frac{1}{e} - \frac{c_0}{\log N} + O\left( \frac{1}{\log^{1+c} N} \right)
  \end{equation}
for some constant $c>0$, where $c_0$ is the explicit constant
\begin{equation}\label{c0-def}
  \begin{split}
  c_0 &\coloneqq \frac{1}{e} \int_0^1 f_e(x)\ dx \\
  &= 0.3044190\dots
\end{split}
\end{equation}
and for any $\alpha>0$, $f_\alpha \colon (0,\infty) \to \R$ denotes the piecewise smooth function
\begin{equation}\label{falpha-def} 
  f_\alpha(x) \coloneqq \left\lfloor \frac{1}{x} \right\rfloor \log \frac{\lceil 1/\alpha x \rceil}{1/\alpha x}.
\end{equation}
In particular, \eqref{t1} and \eqref{Tbound} hold.  In fact the upper bound can be sharpened to
\begin{equation}\label{tna} 
  \frac{t(N)}{N} \leq \frac{1}{e} - \frac{c_0}{\log N} - \frac{c_1+o(1)}{\log^2 N} 
\end{equation}
for an explicit constant $c_1=0.7555\dots$; see 
\end{itemize}
\end{theorem}

For future reference, we observe the simple bounds
\begin{equation}\label{falpha-bound}
 \begin{split}
   0 \leq f_\alpha(x) &\leq \frac{1}{x} \log \frac{1/\alpha x+1}{1/\alpha x}\\
&= \frac{1}{x} \log\left( 1 + \alpha x \right) \\
&\leq \alpha
\end{split}
\end{equation}
for all $x>0$; in particular, $f_\alpha$ is a bounded function.  It however has an oscillating singularity at $x=0$; see \Cref{fig-mean}.

In \Cref{c0-app} we give some details on the numerical computation of the constant $c_0$.

\begin{remark}\label{old} In a previous version \cite{tao} of this manuscript, the weaker bounds
  $$ \frac{1}{e} - \frac{O(1)}{\log N} \leq \frac{t(N)}{N} \leq \frac{1}{e} - \frac{c_0+o(1)}{\log N}$$
were established, which were enough to recover \eqref{t1}, \eqref{Tbound}, and \Cref{main}(i). 
\end{remark}  

\begin{figure}
  \centering
  \includegraphics[width=0.8\textwidth]{integ.png}
  \caption{The piecewise continuous function $x\mapsto \frac{1}{e} f_e(x)$, together with its mean value $c_0 = 0.3044190\dots$ and the upper bound $\frac{\log(1+ex)}{ex}$.  The function exhibits an oscillatory singularity at $x=0$ similar to $\sin \frac{1}{x}$ (but it is always nonnegative and bounded). Informally, the function $f_e$ quantifies the difficulty that large primes in the factorization of $N!$ have in becoming slightly larger than $N/e$ after multiplying by a natural number.}\label{fig-mean}
\end{figure}

As one might expect, the proof of \Cref{main} proceeds by a combination of both theoretical analysis and numerical calculations.  Our main tools to obtain upper and lower bounds on $t(N)$ can be summarized as follows:

\begin{itemize}
  \item In \Cref{greedy-sec}, we discuss \emph{greedy algorithms} to construct subfactorizations, that provide quickly computable, though suboptimal, lower bounds on $t(N)$ for small and medium values;
  \item In \Cref{linprog-sec}, we present a \emph{linear programming} (or \emph{integer programming}) method that provides quite accurate upper and lower bounds on $t(N)$ for small and medium values of $N$;
  \item In \Cref{accounting-sec}, we introduce an \emph{accounting identity} linking the ``$t$-excess'' of a subfactorization with its ``$p$-surpluses'' at various primes, which provides an reasonable upper bound on $t(N)$ for all $N$, and is discussed in more detail in \Cref{accounting-sec};
  \item In \Cref{rearrange-sec}, we extend the \emph{rearrangement approach} from \cite{guy-selfridge} to give a computer-assisted proof that \Cref{main}(iii) holds for sufficiently large $N$.
  \item In \Cref{approx-sec}, we give \emph{modified approximate factorization} strategy, which provides lower bounds on $t(N)$, that become asymptotically quite efficient.
\end{itemize}


The final approach is significantly more complicated than the other three, but gives the most efficient lower bounds in the asymptotic limit $N \to \infty$.  The key idea is to start with an approximate factorization
$$ N! \approx \left(\prod_{j \in I} j\right)^A$$
for some relatively small natural number $A$ (e.g., $A = \lfloor \log^2 N \rfloor$) and a suitable set $I$ of natural numbers greater than or equal to $t$; there is some freedom to select parameters here, and we will take $I$ to be the natural numbers in $(t, t(1+\sigma)]$ that are $3$-rough (coprime to $6$), where $t$ is the target lower bound for $t(N)$ we wish to establish, and $\sigma \coloneqq \frac{3N}{tA}$ is chosen to bring the number of terms in the approximate factorization close to $N$.  With this choice of $I$, this product contains approximately the right number of copies of $p$ for medium-sized primes $p$; but it has the ``wrong'' number of copies of large primes, and is also constructed to avoid the ``tiny'' primes $p=2,3$.  One then performs a number of alterations to this approximate factorization to correct for the ``surpluses'' or ``deficits'' at various primes $p>3$, using the supply of available tiny primes $p=2,3$ as a sort of ``liquidity pool'' to efficiently reallocate primes in the factorization.  A key point will be that the incommensurability of $\log 2$ and $\log 3$ (i.e., the irrationality of $\log 3/\log 2$) means that the $3$-smooth numbers (numbers of the form $2^n 3^m$) are asymptotically dense (in logarithmic scale), allowing for other factors to be exchanged for $3$-smooth factors with little loss\footnote{The weaker results alluded to in \Cref{old} only used the prime $2$ as a supply of ``liquidity'', and thus encountered inefficiencies due to the inability to ``make change'' when approximating another factor by a power of two.}.

\subsection{Author contributions and data}

This project was initially concieved as a single-author manuscript by Terence Tao, but since the release of the initial preprint \cite{tao}, grew to become a collaborative project organized via the Github repository \cite{github}, which also contains the supporting code and data for the project.  The contributions of the individual authors, according to the CRediT categories at \url{https://credit.niso.org/}, are as follows:

{\bf authors should be arranged in alphabetical order of surname.  }

\begin{itemize}
\item Boris Alexeev: ...
\item ...
\item Terence Tao: Conceptualization, Formal Analysis, Methodology, Project Administration, Visualization, Writing -- original draft, Writing -- review \& editing.
\item Markus Uhr: ...
\item Kevin Ventullo: Software
\end{itemize}

\subsection{Acknowledgments}

TT is supported by NSF grant DMS-2347850.  We thank Thomas Bloom for the web site \url{https://www.erdosproblems.com}, where the author learned of this problem, as well as Bryna Kra and Ivan Pan for  corrections.

{\bf list here all contributors to the project who did not wish to be listed as co-authors.}


\section{Notation and basic estimates}

We use the usual asymptotic notation $X = O(Y)$, $X \ll Y$, or $Y \gg X$ to denote an inequality of the form $|X| \leq CY$ for some absolute constant $C$.  We also write $X \asymp Y$ for $X \ll Y \ll X$. For effective estimates, we will use the more precise notation $O_{\leq}(Y)$ to denote any quantity whose magnitude is bounded by exactly at most $Y$. We also use $O_{\leq}(Y)^+$ to denote a quantity of size $O_{\leq}(Y)$ that is also non-negative, that is to say it lies in the interval $[0,Y]$.

If $S$ is a statement, we use $1_S$ to denote its indicator, thus $1_S=1$ when $S$ is true and $1_S=0$ when $S$ is false.  If $x$ is a real number, we use $\lfloor x \rfloor$ to denote the greatest integer less than or equal to $x$, and $\lceil x \rceil$ to be the least integer greater than or equal to $x$.

Throughout this paper, the symbol $p$ (or $p_0$, $p_1$, etc.) is always understood to be restricted to be prime.  The primes $2,3$ will play a special role in this paper and will be referred to as \emph{tiny primes}. 
Call a natural number \emph{$3$-smooth} if it is the product of tiny primes, i.e., it is of the form $2^n 3^m$ for some natural numbers $n,m$, and \emph{$3$-rough} if it is not divisible by any tiny prime, that is to say it is coprime to $6$.  Given a positive real number $x$, we use $\lceil x \rceil^{\langle 2,3 \rangle}$ to denote the smallest $3$-smooth number greater than or equal to $x$.  For instance, $\lceil 5 \rceil^{\langle 2,3 \rangle} = 6$ and $\lceil 10 \rceil^{\langle 2,3 \rangle} = 12$.  

It will be convenient to introduce a variant of this quantity that is close to a power\footnote{The significance of the base $12$ is that the $3$-smooth portion $2^{\nu_2(N!)} 3^{\nu_3(N!)}$ of $N!$, which serves as our ``liquidity pool'', is approximately $2^N 3^{N/2} = 12^{N/2}$; see \eqref{legendre} below.  This makes $\log 12$ a natural ``unit of currency'' in which to conduct various factor exchanges, with various integer linear combinations of $\log 2$ and $\log 3$ usable as ``small change'' to approximate quantities that are not integer multiples of $\log 12$} of $12$.  If $1 \leq L \leq x$ is an additional real parameter, we define
\begin{equation}\label{fancy-kappa-def}
  \lceil x \rceil^{\langle 2,3\rangle}_L \coloneqq 12^a \lceil x/12^a \rceil^{\langle 2,3 \rangle}
\end{equation}
for any real $x \geq L \geq 1$, where $a \coloneqq \lfloor \frac{x/L}{\log 12} \rfloor$ is the largest integer such that $12^a \leq x/L$.  

For any $L \geq 1$, let $\kappa_L$ be the least quantity such that
\begin{equation}\label{kappa-def}  
  x \leq \lceil x \rceil^{\langle 2,3\rangle} \leq \exp(\kappa_L) x 
\end{equation}
holds for all $x \geq L$; see \Cref{fig:nextsmooth}.  In \Cref{power-sec} we establish the following facts:

\begin{figure}
  \centering
  \includegraphics[width=0.8\textwidth]{next_smooth.png}
  \caption{The function $\log \frac{\lceil x \rceil^{\langle 2,3 \rangle}}{x}$, compared against $\kappa_x$. 
  }\label{fig:nextsmooth}
  \end{figure}


\begin{lemma}[Approximation by $3$-smooth numbers]\label{power-lemma}\ 
\begin{itemize}
\item[(i)]  We have $\kappa_{4.5} = \log\frac{4}{3} = 0.28768\dots$ and $\kappa_{40.5} = \log \frac{32}{27} = 0.16989\dots$.
\item[(ii)]  For large $L$, one has $\kappa_L \ll \log^{-c} L$ for some absolute constant $c>0$.
\item[(iii)]  If $1 \leq L \leq x$ are real numbers, then
\begin{equation}\label{mod-kappa}
  x \leq \lceil x \rceil^{\langle 2,3\rangle}_L \leq \exp(\kappa_L) x 
\end{equation}
and for any $0 \leq \gamma < 1$ we have
\begin{equation}\label{12-2}
\frac{\nu_2(\lceil x \rceil^{\langle 2,3\rangle}_L) - 2 \gamma \nu_3(\lceil x \rceil^{\langle 2,3\rangle}_L)}{1-\gamma} \leq \frac{2}{\log 12} \left( \log x  + \kappa^{(2)}_{L,\gamma} \right)
\end{equation}
and
\begin{equation}\label{12-3}
\frac{2\nu_3(\lceil x \rceil^{\langle 2,3\rangle}_L) - \gamma \nu_2(\lceil x \rceil^{\langle 2,3\rangle}_L)}{1-\gamma} \leq \frac{2}{\log 12} \left( \log x + \kappa^{(3)}_{L,\gamma} \right)
\end{equation}
where
\begin{equation}\label{kappastar-2-def}
\kappa^{(2)}_{L,\gamma} \coloneqq \left(\frac{\log 12}{2(1-\gamma)\log 2} - 1\right) \log(12L) + \frac{\kappa_L\log 12}{2 (1-\gamma) \log 2}
\end{equation}
\begin{equation}\label{kappastar-3-def}
  \kappa^{(3)}_{L,\gamma} \coloneqq \left(\frac{\log 12}{(1-\gamma)\log 3} - 1\right) \log(12L) + \frac{\kappa_L\log 12}{(1-\gamma)\log 3}.
\end{equation}
\end{itemize}
\end{lemma}

We remark that when $x$ is a power of $12$, the left-hand sides of \eqref{12-2}, \eqref{12-3} are both equal to $\frac{2}{\log 12} \log x$; thus the estimates \eqref{12-2}, \eqref{12-3} are quite efficient asymptotically.

We use $(a,b)$ to denote the greatest common divisor of $a$ and $b$, $a|b$ to denote the assertion that $a$ divides $b$, and $\pi(x) = \sum_{p \leq x} 1$ to denote the usual prime counting function.

We use $\nu_p(a/b) = \nu_p(a)-\nu_p(b)$ to denote the $p$-adic valuation of a positive natural number $a/b$, that is to say the number of times $p$ divides the numerator $a$, minus the number of times $p$ divides the denominator $b$.  For instance, $\nu_2(32/27)=5$ and $\nu_3(32/27)=-3$. 
If one applies a logarithm to the fundamental theorem of arithmetic, one obtains the identity
\begin{equation}\label{ftoa}
  \sum_p \nu_p(r) \log p = \log r
\end{equation}
for any positive rational $r$.  

For a natural number $n$, we can write
\begin{equation}\label{nup-form} 
  \nu_p(n) = \sum_{j=1}^\infty 1_{p^j|n}.
\end{equation}
Upon taking partial sums, we recover Legendre's formula
\begin{equation}\label{legendre}
  \nu_p(N!) = \sum_{j=1}^\infty \left\lfloor \frac{N}{p^j} \right\rfloor = \frac{N - s_p(N)}{p-1}
\end{equation}
where $s_p(N)$ is the sum of the digits of $N$ in the base $p$ expansion.

Given a putative factorization $\tuple$ of $N!$,  
we refer to the quantity $\nu_p\left( \frac{N!}{\prod \tuple} \right)$ as the \emph{$p$-surplus} of $\tuple$ with respect to the target $N!$, and similarly refer to the negative $-\nu_p\left( \frac{N!}{\prod \tuple} \right) = \nu_p\left( \frac{\prod \tuple}{N!} \right)$ of this surplus as the \emph{$p$-deficit}, with the multiset being \emph{$p$-balanced} if the $p$-surplus (or $p$-deficit) is zero.  Thus, a factorization of $N!$ is achieved if and only if one is balanced at every prime $p$, whereas a subfactorization is achieved if one is either in balance or surplus at every prime $p$.

To bound the factorial, we have the explicit Stirling approximation \cite{robbins}
\begin{equation}\label{stirling}
\log N! = N \log N - N + \log \sqrt{2\pi N} + O_\leq^+\left(\frac{1}{12N}\right),
\end{equation}
valid for all natural numbers $N$. 


We recall the effective prime number theorem from \cite[Corollary 5.2]{dusart}, which asserts that
\begin{equation}\label{pi-lower}
  \pi(x) \geq \frac{x}{\log x} + \frac{x}{\log^2 x}
\end{equation}
for $x \geq 599$ and
\begin{equation}\label{pi-upper}
  \pi(x) \leq \frac{x}{\log x} + \frac{1.2762 x}{\log^2 x}
\end{equation}
for $x >1$.  

We will also need to control sums of somewhat oscillatory functions over primes, for which the bounds in \eqref{pi-lower}, \eqref{pi-upper} are of insufficient strength. Let $y<x$ be real numbers. Given a function $b \colon (y,x] \to \R$, its \emph{total variation}
$\|b\|_{\mathrm{TV}(y,x]}$ is defined as the supremum of the quantities $\sum_{j=0}^{J-1} |b(x_{j+1})-b(x_j)|$ for $y < x_0 \leq \dots \leq x_J \leq x$, and the \emph{augmented total variation} $\|b\|_{\mathrm{TV}^*(y,x]}$ is defined as
$$
\|b\|_{\mathrm{TV}^*(y,x]}
\coloneqq |b(y^+)| + |b(x)| + \|b\|_{\mathrm{TV}(y,x]},$$
$b(y^+) \coloneqq \lim_{t \to y^+} b(t)$ denotes the right limit of $b$ at $y$ (which exists if $b$ is of finite total variation).  Equivalently, $\|b\|_{\mathrm{TV}^*(y,x]}$ is the total variation of $b$ if extended by zero outside of $(y,x]$. The indicator function $1_{(y,x]}$ clearly has an augumented total variation of $2$. 

We will use this augmented total variation to control sums over primes.  More precisely, in \Cref{primes-sec} we will show

\begin{lemma}[Effective bounds for oscillatory sums over primes]\label{osc-lemma}  Let $1423 \leq y \leq x$, and let $b: (y,x] \to \R$ be of bounded total variation.  Then we have the bound
\begin{equation}\label{bv-exact}
    \sum_{y < p \leq x} b(p) \log p = \int_y^x \left(1-\frac{2}{\sqrt{t}}\right) b(t)\ dt + O_{\leq}(\|b\|_{\mathrm{TV}^*(y,x]} E(x))
\end{equation}
where the error function $E(x)$ is defined as
\begin{equation}\label{tilde-e}
  E(x) \coloneqq 0.95 \sqrt{x} + 3.83 \times 10^{-9} x.
\end{equation}
In particular one has
\begin{equation}\label{pix}
  \pi(x) - \pi(y) = \int_y^x \left(1-\frac{2}{\sqrt{t}}\right)\frac{dt}{\log t} + O_\leq\left(2 \frac{E(x)}{\log y}\right).
\end{equation}
If $b$ is non-negative, one also has the upper bound
\begin{equation}\label{bv-upper}
   \sum_{y < p \leq x} b(p) \leq \frac{1}{\log y} \int_y^x b(t)\ dt + \|b\|_{\mathrm{TV}^*(y,x]} \frac{E(x)}{\log y}
\end{equation}
and the lower bound
\begin{equation}\label{bv-lower}
  \sum_{y < p \leq x} b(p) \leq \frac{1-\frac{2}{\sqrt{y}}}{\log x} \int_y^x b(t)\ dt - \|b\|_{\mathrm{TV}^*(y,x]} \frac{E(x)}{\log x}.
\end{equation}
Thus for instance
\begin{equation}\label{pixy-upper}
 \pi(x) - \pi(y) \leq \frac{x-y}{\log y} + 2 \frac{E(x)}{\log y}
\end{equation}
and
\begin{equation}\label{pixy-lower}
  \pi(x) - \pi(y) \geq \left(1-\frac{2}{\sqrt{y}}\right) \frac{x-y}{\log x} - 2 \frac{E(x)}{\log x}.
\end{equation}
One can also replace all occurrences of $E(x)$ here by the classical error term $O(x \exp(-c \sqrt{\log x}))$ for some absolute constant $c>0$ (in which case the $\frac{2}{\sqrt{t}}$ type terms can be absorbed into the error term).
\end{lemma}

We remark that the accuracy in \eqref{bv-exact}, \eqref{pix} in particular is on par with what would be provided by the Riemann hypothesis, as long as $x$ is not too large (e.g., $x \leq 10^{18}$).  The other estimates are not quite as precise, but still adequate for our applications.  The error term $E(x)$ can be improved somewhat for large $x$ (see \eqref{etil-def}), but this simplified version will suffice for our analysis (in particular, the contribution of the second term in \eqref{tilde-e} will be negligible for our applications).  We make the easy remark that $E(x)$ is non-decreasing in $x$, while $E(x)/x$ is non-increasing.

\section{Greedy algorithms}\label{greedy-sec}

The following simple greedy algorithm gives reasonably good performance to obtain large $t$-admissible subfactorizations $\tuple$ of $N!$ for a given choice of $t$ and $N$:

\begin{enumerate}
\item[(0)] Initialize $\tuple$ to be the empty multiset. 
\item[(1)] If $\tuple$ is not a factorization, locate the largest prime $p$ which is currently in surplus: $\nu_p(N!/\prod \tuple) > 0$. 
\item[(2)] If $N! / \prod \tuple$ contains a multiple of $p$ that is greater than or equal to $t$, locate the smallest such multiple, add it to $\tuple$, and return to Step 1.  Otherwise, \texttt{HALT} the algorithm. 
\end{enumerate}

This procedure clearly halts in finite time to produce a $t$-admissible subfactoriation of $N!$.  For instance, applying this procedure with $N=9$, $t=3$ produces the $3$-admissible subfactorization
$$ \{7 \times 1, 5 \times 1, 3 \times 1, 3 \times 1, 3 \times 1, 3 \times 1, 2 \times 2, 2 \times 2, 2 \times 2 \}$$
which recovers the bound $t(9) \geq 3$ from \Cref{nine} (though with a slightly different subfactorization, in which the $8$ is replaced by $4$).

This procedure is efficient for small $N$, for instance attaining the exact value of $t(N)$ for all $N \leq 79$, though it begins to degrade for larger $N$; see \Cref{fig-zoom}.  The performance is also respectable (though not optimal) for medium $N$; for instance, when $N=3 \times 10^5$ and $t=N/3$, it locates a $t$-admissible subfactorization of $N!$ of cardinality $N+372$, which is close to the linear programming limit of $N+455$ established in the next section.

{\bf discuss modifications to the algorithm to make it perform both faster and more accurately}

By using the greedy method, \Cref{main}(ii) can be verified for $N \leq 3 \times 10^5$, and \Cref{main}(iii) can be verified for $8 \times 10^4 \leq N \leq 10^{11}$.    Thus, to resolve these claims, it remains to only establish \Cref{main}(iii) in the regime $43632 \leq N < 8 \times 10^4$ and $N > 10^{11}$, and also to show that this claim fails for $N=43631$.

\section{Linear programming}\label{linprog-sec}

A $t$-admissible subfactorization of $N!$ can also be viewed as a product
$$ \prod_{j \geq t} j^{m_j}$$
for some non-negative integers $m_j$ that obey the linear constraints
\begin{equation}\label{constraints}
  \sum_{j \geq t} m_j \nu_p(j) \leq \nu_p(N!)
\end{equation}
for all primes $p$.  Thus, we have an alternative description of $t(N)$:

\begin{proposition}  For any $N \geq 1$, $t(N)$ is the largest quantity $t$ for which there is a solution to the infinite integer program \eqref{constraints}, where $m_j$ are constrained to be non-negative integers with
$$ \sum_j m_j \leq N.$$
\end{proposition}

This proposition suggests the possibility of using linear programming (or integer programming) methods to provide upper and lower bounds on $t(N)$. An immediate issue for computational purposes is that this program involves an infinite number of variables $m_j$.  However, observe that any $j$ not dividing $N!$ cannot be used; and furthermore if $j$ contains a strictly smaller factor that is at least $t$, then it could be replaced by that factor while still generating a $t$-admissible subfactorization of $N$.  As a consequence of these two facts, $j$ can be restricted to a finite set $J_{t,N}$ of the numbers $t \leq j \leq \max(N,t^2)$ dividing $N!$ with no proper factors greater than or equal to $t$.  This makes the integer program finite (though somewhat large), and in particular computable for small $N$, such as $N \leq 10^4$.  By relaxing the integer program to a linear program, one can also obtain upper and lower bounds as follows:

\begin{itemize}
  \item If one uses linear programming to maximize the quantity $\sum_{j \in J_{t,N}} m_j$ for non-negative \emph{reals} $m_j$ subject to the constraints \eqref{conditions}, and $\sum_{j \in J_{t,N}} \lfloor m_j \rfloor \geq N$, then $\prod_{j \in J_{t,N}} j^{\lfloor m_j \rfloor}$ represents a $t$-admissible subfactorization of $N!$ that witnesses $t(N) \geq t$.
  \item If the dual linear program of finding non-negative reals $w_p$ for each prime $p \leq N$ that satisfy the dual constraints
  \begin{equation}\label{pj}
    \sum_p w_p \nu_p(j) \geq 1
   \end{equation}
   for all $j \in J_{t,N}$ as well as
   \begin{equation}\label{hyp}
    \sum_p w_p \nu_p(N!) < N,
    \end{equation}
    then by multiplying \eqref{constraints} by $w_p$ and summing over $p \leq N$, we see that these constraints force $\sum_{j \in J_{t,N}} m_j < N$, and hence $t(N) < t$.
\end{itemize}

In practice, the linear programming method is extremely accurate in upper and lower bounding $t(N)$; for instance, they give the exact value of $t(N)$ for all $N \leq 600$, with the sole exception of $N=155$, where the linear programming bounds instead give $45 \leq t(155) \leq 46$.  In this case, the (slower) integer programming method can be deployed to verify that $t(155)=45$.


However, the large size of $J_{t,N}$ renders a direct application of the linear programming approach computationally expensive once $N$ exceeds $10^3$ or so.  For the purpose of establishing lower bounsd, one can of course reduce this set arbitrarily; we have found a good choice to be the integers between $t$ and $N$.  For the dual problem, we are also able to make such a reduction, under the additional hypotheses that the weights $w_p$ are non-decreasing:
  
\begin{lemma}[Linear programming bound]\label{lp-upper}  Let $N$ be an natural number and $1 \leq t \leq N/2$.  Suppose for each prime $p \leq N$, one has a non-negative real number $w_p$ which is weakly non-decreasing in $p$ (thus $w_p \leq w_{p'}$ when $p \leq p'$), and such that \eqref{pj} holds
  for all $t \leq j \leq N$, and such that \eqref{hyp} also holds.
\end{lemma}
  
\begin{proof}
By the previous remarks, it will suffice to show that the bound \eqref{pj} in fact holds for all $j \geq t$, not just for $t \leq j \leq N$.  Indeed, if this were not the case, consider the first $j \geq t$ where \eqref{pj} fails.  Take a prime $p$ dividing $j$ and replace it by a prime in the interval $[p/2,p)$ which exists by Bertrand's postulate (or remove $p$ entirely, if $p=2$); this creates a new $j'$ in $[j/2,j)$ which is still at least $t$.  By the weakly decerasing hypothesis on $w_p$, we have
  $$ \sum_p w_p \nu_p(j) \geq \sum_p w_p \nu_p(j')$$
  and hence by the minimality of $j$ we have
  $$ \sum_p w_p \nu_p(j) > 1, $$
  a contradiction.
\end{proof}
  
We have found empirically that using linear programming to maximize the left-hand side of \eqref{hyp} subject to \eqref{pj} for $t \leq j \leq N$ tends to generate weights $w_p$ that are in fact weakly decreasing, so that a rigorous upper bound $t(N) < t$ can be established by this method without needing to expand the linear program to all $j \in J_{t,N}$.  With this technique, the linear programming method can now be run to cover all $N \leq 10^4$ with the exception of $N=155,765,1528,1618,1619,2574,2935,3265,5122,5680,9633$, but in these cases $t(N)$ can be computed exactly by integer programming; see \cite{github} and \Cref{fig-long}.

One can also use this method to accurately bound subfactorizations of $N!$ for larger $N$, although the runtime becomes slow.  For instance, with $N = 3 \times 10^5$ and $t = N/3 = 10^5$, \Cref{lp-upper} can be used to show that any $t$-admissible factorization has cardinality at most $N+455$, while the lower bound linear program produces a $t$-admissible factorization of exactly this cardinality.  This demonstrates \Cref{main}(ii), (iii) for this value of $N$.  The linear programming method can also establish \Cref{main}(iii) in the range $43632 \leq N \leq 8 \times 10^4$, but show that this conjecture fails for $N = 43631$; it also holds for $N=41006$, but fails for all smaller $N$ except for $N=1,2,3,4,5,6,9$.  When combined with the greedy algorithm computations, this resolves \Cref{main}(ii), (iii) except in the asymptotic range $N > 10^{11}$, where it suffices to establish the lower bound $t(N) \geq N/3$.

{\bf more discussion here}


\begin{figure}
  \centering
  \includegraphics[width=0.5\textwidth]{longplot.png}
  \caption{$t(N)/N$ for $80 \leq N \leq 10^4$, obtained via linear programming in most cases (and integer programming in some exceptional cases).  The upper bound from \Cref{upper-crit} is surprisingly sharp, as is the refined asymptotic $1/e - c_0/\log N - c_1/\log^2 N$, though the cruder asymptotics $1/e$ or $1/e - c_0/\log N$ are significantly poorer approximations.}
  \label{fig-long}
  \end{figure}

\section{The accounting identity}\label{accounting-sec}

Given a $t$-admissible multiset $\tuple$ (which we view as an approximate factorization of $N!$), we can apply the fundamental theorem of arithmetic \eqref{ftoa} to the rational number  $N!/\prod \tuple$ and rearrange to obtain the \emph{accounting identity}
\begin{equation}\label{accounting} 
  \excess_t(\tuple) + \sum_p \nu_p\left( \frac{N!}{\prod \tuple} \right) \log p = \log N! - |\tuple| \log t
\end{equation}
where we define the \emph{$t$-excess} $\excess_t(\tuple)$ of the multiset $\tuple$ by the formula
\begin{equation}\label{excess-def}
  \excess_t(\tuple) \coloneqq \sum_{a \in \tuple} \log \frac{a}{t}.
\end{equation}

\begin{example} Suppose one wishes to factorize $5! = 2^3 \times 3 \times 5$.  The attempted $3$-admissible factorization $\tuple \coloneqq \{3,4,5,5\}$ has a $2$-surplus of $\nu_2(5!/\prod \tuple) = 1$, is in balance at $3$, and has a $5$-deficit of $\nu_2(\prod \tuple/5!) = 1$, so it is not a factorization or subfactorization of $5!$.  The $3$-excess of this multiset is
  $$ \excess_3(\tuple) = \log \frac{3}{3} + \log \frac{4}{3} + \log \frac{5}{3} + \log \frac{5}{3} = 1.3093\dots$$
  and the accounting identity \eqref{accounting} become
  $$ 1.3093\dots + \log 2 - \log 5 = 0.3930\dots = \log 5! - 4 \log 3.$$
  If one replaces one of the copies of $5$ in ${\mathcal B}$ with a $2$, this erases both the $2$-surplus and the $5$-deficit, and creates a factorization $\tuple' = \{2,3,4,5\}$ of $5!$; the $3$-excess now drops to
  $$ \excess_3(\tuple) = \log \frac{2}{3} + \log \frac{3}{3} + \log \frac{4}{3} + \log \frac{5}{3}  = 0.3930\dots,$$
  bringing the accounting identity back into balance.
\end{example}
  
In view of \Cref{subfac}, one can now equivalently describe 
$t(N)$ as follows:

\begin{lemma}[Equivalent description of $t(N)$]\label{t-descrip}  $t(N)$ is the largest quantity $t$ for which there exists a $t$-admissible subfactorization of $N!$ with
$$ \excess_t(\tuple) + \sum_p \nu_p\left( \frac{N!}{\prod \tuple} \right) \log p \leq \log N! - N \log t.$$
\end{lemma}

One can view $\log N! - N\log t$ as an available ``budget'' that one can ``spend'' on some combination of $t$-excess and $p$-surpluses.  For $t$ of the form $t = N/e^{1+\delta}$ for some $\delta>0$, the budget can be computed using the Stirling approximation \eqref{stirling} to be $\delta N + O(\log N)$.  The non-negativity of the $t$-excess and $p$-surpluses recovers the trivial upper bound \eqref{trivial}; but one can improve upon this bound by observing that large prime factors of $N!$ inevitably generate a noticeable $t$-excess, as follows.  

\begin{lemma}[Upper bound criterion]\label{upper-crit}  Suppose that $1 \leq t \leq N$ are such that
  \begin{equation}\label{contra}
     \sum_{\frac{t}{\lfloor\sqrt{t}\rfloor} < p \leq N} f_{N/t}(p/N) > \log N! - N \log t,
  \end{equation}
  where $f_{N/t}$ was defined in \eqref{falpha-def}.
  Then $t(N) < t$.
  \end{lemma}

  \begin{proof} Suppose for contradiction that $t(N) \geq t$, then we can find a $t$-admissible factorization $\tuple$ of $N!$.  The accounting identity then gives
  \begin{equation}\label{ai}
  \sum_{a \in \tuple} \log \frac{a}{t} = \excess_t(\tuple) = \log N! - N \log t.
  \end{equation}
  We write $f_{N/t}(p/N) = \lfloor \frac{N}{p} \rfloor g_t(p)$, where $g_t(p) \coloneqq \log (\frac{p}{t} \lceil \frac{t}{p} \rceil)$.  We claim that 
    \begin{equation}\label{ai2}
      \log \frac{a}{t} \geq g_t(p_{a,1}) + \dots + g_t(p_{a,k_a})
    \end{equation}
for all $a \in \tuple$, where $p_{a,1},\dots,p_{a,k_a}$ are the primes greater than $\frac{t}{\lfloor \sqrt{t}\rfloor}$ that divide $a$ (counting multiplicity).  For $k_a=0$ this is clear since $a \geq t$.  For $k_a=1$, we can write $a = d_a p_{a,1}$ where $p_{a,1} > \frac{t}{\sqrt{t}+1}$ and $d_a \geq \lceil \frac{t}{p_{a,1}} \rceil$, so that
    $$ \log \frac{a}{t} = \log \left(\frac{p_{a,1}}{t}d_a \right) \geq g_t(p_{a,1}),$$
    again giving \eqref{ai2}.  For $k_a \geq 2$, we have $a \geq p_{a,1} \dots p_{a,k}$, hence
    \begin{align*}
      \log \frac{a}{t} - \sum_{j=1}^{k_a} g_t(p_{a,j}) &\geq \sum_{j=1}^{k_a} (\log p_{a,j} - g_t(p_{a,j})) - \log t \\
      &= \sum_{j=1}^{k_a} \left(\log t - \log \left \lceil \frac{t}{p_{a,j}} \right\rceil \right) - \log t \\
      &\geq \sum_{j=1}^{k_a} \left(\log t - \log \sqrt{t} \right) - \log t \\
      &\geq 0
    \end{align*}
    which again gives \eqref{ai2}.  Summing \eqref{ai2} over all $a \in \tuple$ and inserting into \eqref{ai}, we conclude that
    $$ \sum_{p > \frac{t}{\lfloor\sqrt{t}\rfloor}} \nu_p(N!) g_t(p) \leq \log N! - N \log t.$$    
    By \eqref{legendre}, we can bound $\nu_p(N!) g_t(p)$ by $\lfloor N/p \rfloor g_t(p) = f_{N/t}(p/N)$.  This contradicts \eqref{contra}, giving the claim. 
  \end{proof}

In practice, \Cref{upper-crit} gives quite good upper bounds on $N$, especially when $N$ is large, although for medium $N$ the linear programming method is superior: see \Cref{fig1}, \Cref{fig1-alt}, \Cref{fig-zoom}.
    
\begin{figure}
  \centering
  \includegraphics[width=0.8\textwidth]{newplot_600.png}
  \caption{An enlarged version of \Cref{fig1-alt}, displaying the lower bound from the greedy algorithm and the upper bound from \Cref{upper-crit}.  The linear programming upper and lower bounds are exact in this region, except for $N=155$ in which the upper bound is off by one.}\label{fig-zoom}
\end{figure}


We can now prove the upper bound portion of \Cref{main}(iv):

  \begin{proposition}\label{upper-bound}  For large $N$, one has
    $$ \frac{t(N)}{N} \leq \frac{1}{e} - \frac{c_0}{\log N} - \frac{c_1 + o(1)}{\log^2 N}$$
    where
    \begin{align*}
      c'_1 &\coloneqq \frac{1}{e} \int_0^1 f_e(x) \log \frac{1}{x}\ dx \approx 0.3702\\
      c''_1 &\coloneqq \sum_{k=1}^\infty \frac{1}{k}  \log\left( \frac{e}{k} \left\lceil \frac{k}{e} \right\rceil \right) \approx 1.6796\\
      c_1 &\coloneqq c'_1 + c_0 c''_1 - ec_0^2/2 \approx 0.7555.
    \end{align*}
  \end{proposition}
    
    \begin{proof}  We apply \Cref{upper-crit} with
      $$ t \coloneqq \frac{1}{e} - \frac{c_0}{\log N} - \frac{c_1-\eps}{\log^2 N}$$
    for a small absolute constant $\eps>0$.  From Taylor expansion and the Stirling approximation one sees that
    $$ \log N! - N \log t = ec_0 \frac{N}{\log N} + (ec_1 - \frac{1}{2} e^2 c_0^2 - e\eps+o(1)) \frac{N}{\log^2 N}$$
    so it will suffice to establish the lower bound
    \begin{equation}\label{targ-sum}
      \sum_{\frac{t}{\lfloor\sqrt{t}\rfloor} < p \leq N} f_{N/t}(p/N) \geq ec_0 \frac{N}{\log N} + (ec_1 - \frac{1}{2} e^2 c_0^2 - e \eps + o(1)) \frac{N}{\log^2 N}
    \end{equation}
    for $N$ sufficiently large depending on $\eps$.

    For $N$ large enough, we have $\frac{t}{\lfloor\sqrt{t}\rfloor} \leq \frac{N}{\log^3 N}$.
    On the interval $[1/\log^3 N,1]$, the piecewise smooth function $f_{N/t}$ is bounded by $O(1)$ thanks to \eqref{falpha-bound}, and has a total variation of $O(\log^3 N)$; the same is then true for the rescaled function $x \mapsto f_{N/t}(x/N)$ on $[N/\log^3 N,1]$.  By \Cref{osc-lemma} (with classical error term), the left-hand side of \eqref{targ-sum} is at least
    $$ \int_{N/\log^3 N}^N f_{N/t}(x/N) \frac{dx}{\log x} + O\left( N \exp(-c\sqrt{\log N}) \right)$$
    for some $c>0$.  Performing a change of variable, we reduce to showing that
    $$  \int_{1/\log^3 N}^1 f_{N/t}(x) \frac{\log N}{\log(Nx)}\ dx 
    \geq  ec_0 + \frac{ec_1 - \frac{1}{2} e^2 c_0^2 - e \eps + o(1)}{\log N}.$$
    By Taylor expansion, we have
    $$ \frac{\log N}{\log(Nx)} = 1 + \frac{1}{\log N} \log \frac{1}{x} + o(\frac{1}{\log N})$$
    and from dominated convergence we have
  $$
  \int_{1/\log^3 N}^1 f_{N/t}(x) \log \frac{1}{x}\ dx = ec'_1 + o(1)$$
and hence by definition of $c_1$, we reduce to showing that
$$  \int_{1/\log^3 N}^1 f_{N/t}(x)\ dx 
\geq  ec_0 + \frac{ec_0 c''_1 - e^2 c_0^2 - e\eps + o(1)}{\log N}.$$
By performing a rescaling by $N/et = 1 + \frac{ec_0+o(1)}{\log N}$, the left-hand side may be written as
$$ \left(1 - \frac{ec_0+o(1)}{\log N}\right) \int_{N/et\log^3 N}^{N/et}
\left\lfloor \frac{N/et}{x} \right\rfloor \log\left(ex \left\lceil \frac{1}{ex} \right\rceil \right) $$
so it will suffice to show that
$$
\int_{N/et\log^3 N}^{N/et}
\left\lfloor \frac{N/et}{x} \right\rfloor \log\left(ex \left\lceil \frac{1}{ex} \right\rceil \right)\ dx \geq ec_0 + \frac{ec_0 c''_1 - e\eps + o(1)}{\log N}.$$
Since
$$ \int_{1/\log^2 N} \lfloor \frac{N/et}{x} \rfloor \log\left(ex \left\lceil \frac{1}{ex} \right\rceil \right) = ec_0 - \frac{o(1)}{\log N}$$
it suffices to show that
$$
\int_{1/log^2 N}^{N/et}
(\left\lfloor \frac{N/et}{x} \right\rfloor - \left\lfloor \frac{1}{x} \right\rfloor) \log\left(ex \left\lceil \frac{1}{ex} \right\rceil \right)\ dx \geq \frac{ec_0 c'' - e\eps + o(1)}{\log N}.$$
Let $K$ be sufficiently large depending on $\eps$, then for $N$ sufficiently large depending on $K$ we can lower bound the left-hand side by
$$ \sum_{k=1}^K \int_{1/k}^{N/etk} \log\left(ex \left\lceil \frac{1}{ex} \right\rceil \right)\ dx;$$
since $\frac{N}{etk} = \frac{1}{k} + \frac{ec_0}{k \log N}$, we can lower bound this (using the irrationality of $e$) by
$$ \frac{ec_0+o(1)}{\log N} \sum_{k=1}^K \frac{1}{k} \log\left(\frac{e}{k} \lceil \frac{k}{e} \rceil\right)$$
for sufficiently large $N$.  Since the sum here can be made arbitrarily close to $c''_0$ by increasing $K$, we obtain the claim.
\end{proof}
    
We can now establish \Cref{main}(i): 

\begin{proposition}\label{tne} One has $t(N)/N < 1/e$ for $N \neq 1,2,4$.
\end{proposition}

\begin{proof}  From existing data on $t(N)$ (or the linear programming method) one can verify this claim for $N < 80$ (see \Cref{fig1}), so we assume that $N\geq 80$.
  
Applying \Cref{upper-crit}, \eqref{stirling}, it suffices to show that
\begin{equation}\label{base-test-ineq}
   \sum_{p \geq \frac{N/e}{\lfloor\sqrt{N/e}\rfloor}} f_{e}(p/N) > \frac{1}{2} \log(2\pi N) + \frac{1}{12N}.
\end{equation}
This may be easily verified numerically in the range $80 \leq N \leq 5000$ (see \Cref{fig2}).
We will discard the $\lfloor\sqrt{N/e}\rfloor$ denominator, and reduce to showing
\begin{equation}\label{test-ineq}
  \sum_{N/e < p \leq N} f_{e}(p/N) > \frac{1}{2} \log(2\pi N) + \frac{1}{12N}
\end{equation}
for $N > 5000$.  On $[1/e,1]$, one can compute
$$ \|f_e\|_{\mathrm{TV}^*(1/e,1]}
= 4 - 2 \log 2$$
so by \Cref{osc-lemma} (noting that $5000 > 1423e$) we have
$$ \sum_{N/e < p \leq N} f_{e}(p/N) \log p
\geq N \int_{1/e}^1 \left(1-\frac{2}{\sqrt{Nx}}\right) f_e(x)\ dx - (4 - 2 \log 2) E(N).
$$
By upper bounding $\log p$ by $\log N$ and lower bounding $\left(1-\frac{2}{\sqrt{Nx}}\right)$ by $1 - \frac{2}{\sqrt{N/e}}$, it suffices to show that
$$ \left(1 - \frac{2}{\sqrt{N/e}}\right) \int_{1/e}^1 f_e(x)\ dx \geq
(4 - 2 \log 2) \frac{E(N)}{N}
+ \frac{\log N \log(2\pi N)}{2N} + \frac{\log N}{12N^2},$$
which is easily verified for $N \geq 5000$ (one has $\int_{1/e}^1 f_e(x)\ dx = \frac{2}{e}-\frac{\log 2}{2}=0.3891\dots$ and $4-2\log 2 = 2.613\dots$, while $E(N)/N \leq 0.015$, and the other two terms on the right-hand side are negligible).
\end{proof}

\begin{figure}
  \centering
  \includegraphics[width=0.8\textwidth]{lhs_rhs.png}
  \caption{A plot of the left and right-hand sides of \eqref{base-test-ineq}, \eqref{test-ineq} for $80 \leq N < 5000$.}\label{fig2}
\end{figure}

\section{Rearranging the standard factorization}\label{rearrange-sec}

In this section we describe an approach to establishing lower bounds on $t(N)$ by starting with the standard factorization $\{1,\dots,N\}$, dividing out some small prime factors from some of the terms, and then redistributing them to other terms.  This approach was introduced in \cite{guy} to give lower bounds of the shape $\frac{t(N)}{N} \geq \frac{3}{16} + o(1)$ (by redistributing powers of two only) and $\frac{t(N)}{N} \geq \frac{1}{4} + o(1)$.  With computer assistance, we are also able to show that $\frac{t(N)}{N} \geq \frac{1}{3}+o(1)$ for sufficiently large $N$, in a simpler fashion than the method used to prove \Cref{main}(iv) in the next section.

\textbf{details needed here}


\section{Modified approximate factorizations}\label{approx-sec}

In this section we present and then analyze an algorithm that starts with an \emph{approximate} factorization $\tuple^{(0)}$ of $N!$, which is $t$-admissible but omits all tiny primes, and is approximately in balance in small and medium primes, and attempts to ``repair'' this factorization to establish a lower bound of the form $t(N) \geq t$.  

To describe the criterion for the algorithm to succeed, it will be convenient to introduce the following notation.
For $a_+,a_- \in [0,+\infty]$, we define the asymmetric norm $|x|_{a_+,a_-}$ of a real number $x$ by the formula
$$ 
|x|_{a_+,a_-} \coloneqq  \begin{cases} 
  a_+ |x| & x\geq 0 \\
  a_- |x| & x\leq 0,
\end{cases}
$$
with the usual convention $+\infty \times 0 = 0$.
If $a_+,a_-$ are finite, this function is Lipschitz with constant $\max(a_+,a_-)$.  One can think of $a_+$ as the ``cost'' of making $x$ positive, and $a_-$ as the
``cost'' of making $x$ negative. 

The analysis of the algorithm is now captured by the following proposition.

\begin{proposition}[Repairing an approximate factorization]\label{repair}  Let $N, K$ be natural numbers, and let $1 \leq t \leq N$ be an additional parameter obeying the conditions
\begin{equation}\label{conditions}
    \frac{t}{K} \geq \sqrt{N}; \quad \frac{t}{K^2} \geq K \geq 5.
\end{equation}
We also assume that there are additional parameters $\kappa_* > 0$ and $0 \leq \gamma_2, \gamma_3 < 1$, such that there exist $3$-smooth numbers
\begin{equation}\label{tlip} 
  t \leq 2^{n_2} 3^{m_2}, 2^{n_3} 3^{m_3} \leq e^{\kappa_*} t
\end{equation}
such that
\begin{equation}\label{nm}
  2m_2 \leq \gamma_2 n_2; \quad n_3 \leq 2\gamma_3 m_3.
\end{equation}
We define the ``norm''
$$ \| (n,m) \|_\gamma \coloneqq \max\left( \frac{n-2\gamma_2 m}{1-\gamma_2}, \frac{2m-\gamma_3 n}{1-\gamma_3} \right).$$
Let $\tuple^{(0)}$ be a $t$-admissible multiset of natural numbers, with all elements of $\tuple^{(0)}$ at most $(t/K)^2$, and suppose that one has the inequalities
\begin{equation}\label{delta-cond}
\sum_{i=1}^8 \delta_i \leq \delta
\end{equation}
and
\begin{equation}\label{alpha-cond}
  \sum_{i=1}^7 \alpha_i \leq 1
\end{equation} 
where
\begin{align}
\delta_1 &\coloneqq \frac{1}{N} \excess_t(\tuple^{(1)}) \label{delta1-def}  \\
\delta_2 &\coloneqq \frac{1}{N} \sum_{t/K < p \leq N} f_{N/t}(p/N) \label{delta2-def}  \\
\delta_3 &\coloneqq \frac{\kappa_{4.5}}{N}  \sum_{3 < p_1 \leq t/K} \left|\nu_{p_1}\left( \frac{N!}{\prod \tuple^{(0)}} \right)\right| \label{delta3-def}  \\
\delta_4 &\coloneqq \kappa_{4.5} \sum_{K < p_1 \leq t/K} A_{p_1} \label{delta4-def}  \\
\delta_5 &\coloneqq \kappa_{4.5} \sum_{3 < p_1 \leq K} |A_{p_1} - B_{p_1}|_{\frac{\log p_1}{\log(t/K^2)},1} \label{delta5-def}  \\
\delta_6 &\coloneqq \frac{\kappa_{4.5}}{N} \label{delta6-def}  \\
\delta_7 &\coloneqq \frac{\kappa_*}{\log t} \left( \frac{\log 12}{2} - B_2 \log 2 - B_3 \log 3\right)\label{delta7-def}  \\
\delta_8 &\coloneqq \frac{2(\log t + \kappa_*)}{N} \label{delta8-def}\\
\delta &\coloneqq \frac{1}{N} \log N! - \log t\label{delta-def}  
\end{align}
\begin{align}
\alpha_1 &\coloneqq \frac{1}{N} \left\| \left( \nu_{2}\left(\prod \tuple^{(0)}\right), \nu_{3}\left(\prod \tuple^{(0)}\right)\right) \right\|_\gamma \label{alpha1-def}  \\
\alpha_2 &\coloneqq \left\| (B_2, B_3) \right\|_\gamma \label{alpha2-def}  \\
\alpha_3 &\coloneqq \frac{2}{N\log 12} \left(\log \frac{t}{K} + \kappa_{**}\right) \sum_{3 < p_1 \leq t/K} \left|\nu_{p_1}\left( \frac{N!}{\prod \tuple^{(0)}} \right)\right| \label{alpha3-def}  \\
\alpha_4 &\coloneqq \frac{2}{\log 12} \sum_{K < p_1 \leq t/K} \left(\log \frac{t}{p_1} + \kappa_{**}\right) A_{p_1} \label{alpha4-def}  \\
\alpha_5 &\coloneqq \frac{2}{\log 12} \sum_{3 < p_1 \leq K} \left|A_{p_1} - B_{p_1}\right|_{\frac{\log p_1}{\log(t/K^2)} (\log K^2 + \kappa_{**}), \log p_1 + \kappa_{**}} \label{alpha5-def}  \\
\alpha_6 &\coloneqq \frac{2}{N\log 12} (\log t + \kappa_{**}) \label{alpha6-def}  \\
\alpha_7 &\coloneqq \max\left( \frac{\log(2N)}{(1-\gamma_2)N\log 2},  \frac{2\log(3N)}{(1-\gamma_3)N\log 3}\right)\label{alpha7-def}  
\end{align}
\begin{align}
\kappa_{**} &\coloneqq \max(\kappa^{(2)}_{4.5, \gamma_2}, \kappa^{(3)}_{4.5, \gamma_3}) \label{kappastar-def}  \\
A_{p_1} &\coloneqq \frac{1}{N} \sum_m \nu_{p_1}(m) | \{ a \in \tuple^{(0)}: a = mp \hbox{ for a prime } p > t/K \}| \label{A-def} \\
B_{p_1} &\coloneqq \frac{1}{N} \sum_{m \leq K} \nu_{p_1}(m) \sum_{\frac{t}{m} \leq p < \frac{t}{m-1}} \left \lfloor \frac{N}{p} \right\rfloor,\label{B-def} 
\end{align}
with the convention that the upper bound $p < \frac{t}{m-1}$ in \eqref{B-def} is vacuous when $m=1$.  Then $t(N) \geq t$.
\end{proposition}

Note that the quantities $\delta_2, \delta$ essentially appeared previously \Cref{upper-crit}.  As we shall see, when $N$ is large we will be able to construct multisets $\tuple^{(0)}$ that make all the other quantities $\delta_i, i \neq 2$ and $\alpha_i$ quite small, which is how we shall establish such results as \Cref{main}(iii) and the lower bound in \Cref{main}(iv).

In practice, the parameter $K$ will be quite small compared to $N$, and the quantities $\gamma_2, \gamma_3, \kappa_{*}$ will also be somewhat smaller than $1$.

The rest of this section will be devoted to the proof of this proposition.  It will be convenient to divide the primes into four classes:
  \begin{itemize}
    \item \emph{Tiny primes} $p=2,3$.
    \item \emph{Small primes} $3 < p \leq K$.
    \item \emph{Medium primes} $K < p \leq t/K$.
    \item \emph{Large primes} $p > t/K$.
    \end{itemize}
Initially, the multiset $\tuple^{(0)}$ may have the ``wrong'' number of factors at large primes.  We fix this by applying the following modifications to $\tuple^{(0)}$:
\begin{itemize}
  \item[(a)] Remove all elements of $\tuple^{(0)}$ that are divisible by a large prime $p > t/K$ from the multiset.
  \item[(b)] For each large prime $p > t/K$, add $\nu_p(N!)$ copies of $p \lceil t/p \rceil$ to the multiset.
\end{itemize}
We let $\tuple^{(1)}$ be the multiset formed after completing both Step (a) and Step (b).  We make two simple observations:
\begin{itemize}
\item[(A)] Since the elements of $\tuple^{(0)}$ are at most $(t/K)^2$, all the elements removed in Step (a) are of the form $mp$ where $m \leq t/K$.
\item[(B)] For each large prime $p$ considered in Step (b),  one has $\nu_p(N!) = \lfloor N/p \rfloor$ by \eqref{legendre} and \eqref{conditions}, while $\lceil t/p \rceil \leq K \leq t/K$ (again by \eqref{conditions}).  
\end{itemize}
From this, we see that $\tuple^{(1)}$ is automatically $t$-admissible, and in balance at any large prime $p > t/K$:
$$ \nu_p\left(\frac{N!}{\prod \tuple^{(1)}}\right) = 0.$$
For medium primes $K < p_1 \leq t/K$, one can have some increase in the $p_1$-surplus coming from Step (a), which is described by \eqref{A-def}:
$$ \nu_{p_1}\left(\frac{N!}{\prod \tuple^{(1)}}\right) = \nu_{p_1}\left(\frac{N!}{\prod \tuple^{(0)}}\right) + NA_{p_1}.$$
For small or tiny primes $p \leq K$, one also has some possible decrease in the $p_1$-surplus coming from Step (b), which is described by \eqref{B-def}:
$$ \nu_{p_1}\left(\frac{N!}{\prod \tuple^{(1)}}\right) = \nu_{p_1}\left(\frac{N!}{\prod \tuple^{(0)}}\right) + N(A_{p_1} - B_{p_1}).$$
In particular, we have from \eqref{alpha1-def}, \eqref{alpha2-def} and the triangle inequality that
\begin{equation}\label{alpha-1}
\frac{1}{N} \left\| \nu_2\left(\prod \tuple^{(1)}\right), \nu_3\left(\prod \tuple^{(1)}\right)\right\|_\gamma \leq \alpha_1 + \alpha_2.
\end{equation}

Each element removed in Step (a) reduces the $t$-excess, while each element $p \lceil t/p \rceil$ added in Step (b) increases the $t$-excess by $\log \frac{\lceil t/p \rceil}{t/p}$, so each large prime $t/K < p \leq N$ contributes a net of $\lfloor \frac{N}{p} \rfloor \log \frac{\lceil t/p \rceil}{t/p} = f_{N/t}(p/N)$ to the $t$-excess.  Thus by \eqref{delta1-def}, \eqref{delta2-def} we have
\begin{equation}\label{excess-1} 
  \frac{1}{N} \excess_t(\tuple^{(1)}) \leq \delta_1 + \delta_2.
\end{equation}

Now we bring the multiset $\tuple^{(1)}$ into balance at small and medium primes $3 < p \leq t/K$.  We make the following observations:
\begin{itemize}
\item[(C)]  If an element in $\tuple^{(1)}$ is divisible by some small or medium prime $3 < p \leq t/K$, and one replaces $p$ by $\lceil p \rceil^{\langle 2,3 \rangle}_{4.5}$ in the factorization of that element, then the $p$-deficit decreases by one, while (by \Cref{power-lemma}) the $t$-excess increases by at most $\kappa_{4.5}$, and the quantity $\| (\nu_2(\prod \tuple^{(1)}),\nu_3(\prod \tuple^{(1)}))\|_\gamma$ increases by at most $\frac{2}{\log 12}(\log p + \kappa_{**})$.  All other $p_1$-surpluses or $p_1$-deficits for $p_1 \neq 2,3,p$ remain unaffected.
\item[(D)]  If one adds an element of the form $m \lceil t/m \rceil^{\langle 2,3 \rangle}_{4.5}$ to $\tuple^{(1)}$ for some $m \leq t/K$ that is the product of small or medium primes $3 < p \leq t/K$, then the $p$-surpluses at small or medium primes $p$ decrease by $\nu_p(m)$, while (by \Cref{power-lemma}) the $t$-excess increases by at most $\kappa_{4.5}$, and the quantity $\sup_{p_0=2,3} \frac{p_0-1}{N} \nu_{p_0}\left(\frac{N!}{\prod \tuple^{(1)}}\right)$ increases by at most $\frac{2}{\log 12}(\log(t/m) + \kappa_{**})$.  The $p$-surpluses or $p$-deficits at medium or large primes remain unaffected.
\end{itemize}

With these observations in mind, we perform the following modifications to the multiset $\tuple^{(1)}$.
\begin{itemize}
\item[(c)] If there is a $p_1$-deficit $\nu_{p_1}(\prod \tuple^{(1)}/N!) > 0$ at some small or medium prime $3 < p_1 \leq t/K$, then we perform the replacement of $p_1$ in one of the elements of $\tuple^{(1)}$ with $\lceil p_1 \rceil^{\langle 2,3\rangle}_{4.5}$ as per observation (C), repeated $\nu_{p_1}(\prod \tuple^{(1)}/N!)$ times, in order to eliminate all such deficits.
\item[(d)] If there is a $p$-surplus $\nu_p(\prod N!/\tuple^{(1)}) > 0$ at some medium prime $K < p \leq t/K$, we add the element 
$p \lceil t/p \rceil^{\langle 2,3 \rangle}_{4.5}$ to $\tuple^{(1)}$ as per observation (D), $\nu_p(\prod N!/\tuple^{(1)})$ times, in order to eliminate all such surpluses at medium primes.
\item[(d')] If there are $p$-surpluses $\nu_p(\prod N!/\tuple^{(1)}) > 0$ at some small primes $3 < p \leq K$, we multiply all these primes together, then apply the greedy algorithm to factor them into products $m$ in the range $t/K^2 < m \leq t/K$, plus at most one exceptional product in the range $1 < m \leq t/K$.  For each of these $m$, add $m \lceil t/m \rceil^{\langle 2,3 \rangle}_{4.5}$ to $\tuple^{(1)}$ as per observation (D), to eliminate all such surpluses at small primes.
\end{itemize}
Call the multiset formed from $\tuple^{(1)}$ formed as the outcome of applying Steps (c), (d), (d') as $\tuple^{(2)}$.  The product of all the primes arising in Step (d') has logarithm equal to
$$ \sum_{3 < p_1 \leq K} \left|\nu_{p_1}\left( \frac{N!}{\prod \tuple^{(1)}} \right) \right|_{\log p_1,0} = \sum_{3 < p_1 \leq K} \left|\nu_p\left( \frac{N!}{\prod \tuple^{(0)}} \right) \right|_{\log p_1,0}$$
and hence the number of non-exceptional $m$ arising in (d') is at most
$$ \sum_{3 < p_1 \leq K} \left|\nu_p\left( \frac{N!}{\prod \tuple^{(0)}} \right) \right|_{\frac{\log p_1}{\log(t/K^2)},0}.$$
The total excess of $\tuple^{(2)}$ is increased in Step (c) by at most
$$ \kappa_{4.5} \sum_{3 < p_1 \leq t/K} \left|\nu_{p_1}\left( \frac{N!}{\prod \tuple^{(1)}} \right) \right|_{0,1}
=  \kappa_{4.5} \sum_{3 < p_1 \leq t/K} \left|\nu_{p_1}\left( \frac{N!}{\prod \tuple^{(0)}} \right) + N(A_{p_1} - B_{p_1}) \right|_{0,1},$$
in Step (d) by at most
$$ \kappa_{4.5} \sum_{K < p_1 \leq t/K} \left|\nu_{p_1}\left( \frac{N!}{\prod \tuple^{(1)}} \right) \right|_{1,0}
=  \kappa_{4.5} \sum_{K < p_1 \leq t/K} \left|\nu_{p_1}\left( \frac{N!}{\prod \tuple^{(0)}} \right) + NA_{p_1} \right|_{1,0},$$
and in Step (c) by at most
$$ \kappa_{4.5} \left(1 + \sum_{3 < p_1 \leq K} \left|\nu_{p_1}\left( \frac{N!}{\prod \tuple^{(0)}} \right) \right|_{\frac{\log p_1}{\log(t/K^2)},0}\right).$$
From the triangle inequality and \eqref{excess-1}, \eqref{delta3-def}, \eqref{delta4-def}, \eqref{delta5-def}, \eqref{delta6-def}, we then have
\begin{equation}\label{excess-2}
   \frac{1}{N} \excess_t(\tuple^{(2)}) \leq \sum_{i=1}^6 \delta_i.
\end{equation}
Similarly, the quantity $\frac{1}{N} \| (\nu_2(\prod \tuple^{(1)}),\nu_3(\prod \tuple^{(1)}))\|_\gamma$ is increased in Step (c) by at most
$$\frac{2}{N\log 12} \sum_{3 < p_1 \leq t/K} \left|\nu_{p_1}\left( \frac{N!}{\prod \tuple^{(0)}} \right) + N(A_{p_1} - B_{p_1}) \right|_{0,\log p_1 + \kappa_{**}},$$
in Step (d) by at most
$$\frac{2}{N\log 12} \sum_{K < p_1 \leq t/K} \left|\nu_{p_1}\left( \frac{N!}{\prod \tuple^{(0)}} \right) + NA_{p_1} \right|_{\log(t/p_1) + \kappa_{**},0},$$
and in Step (d') by at most the sum of
$$\frac{2}{N\log 12} \sum_{3 < p_1 \leq K} \left|\nu_{p_1}\left( \frac{N!}{\prod \tuple^{(0)}} \right) + N(A_{p_1}-B_{p_1}) \right|_{\log(K^2) + \kappa_{**},0}$$
and
$$\frac{2}{N\log 12} \left( \log t + \kappa_{**} \right)$$
so by \eqref{alpha-1}, \eqref{alpha3-def}, \eqref{alpha4-def}, \eqref{alpha5-def}, \eqref{alpha6-def}, and the triangle inequality we have
\begin{equation}\label{alpha-2} 
\frac{1}{N} \| (\nu_2(\prod \tuple^{(2)}), \nu_3(\prod \tuple^{(3)})) \|_\gamma \leq \sum_{i=1}^6 \alpha_i.
\end{equation}

By construction, the tuple $\tuple^{(2)}$ is $t$-admissible, and in balance at all small, medium, and large primes $p > 3$; thus $N!/\prod \tuple^{(2)} = 2^n 3^m$ for some integers $n,m$.  From \eqref{alpha-2}, \eqref{alpha-cond}, \eqref{legendre}, \eqref{alpha7-def} we have
\begin{align*}
n - 2\gamma_2 m &= \nu_2(N!) - 2\gamma_2 \nu_3(N!) - \left(\nu_2\left(\prod \tuple^{(2)}\right) - 2\gamma_2 \nu_3\left(\prod \tuple^{(2)}\right)\right) \\
&\geq \nu_2(N!) - 2\gamma_2 \nu_3(N!) - N (1-\gamma_2) \sum_{i=1}^6 \alpha_i \\
&> N - 
 \frac{\log N}{\log 2} - 1 - \gamma_2 N - N (1-\gamma_2) (1 - \alpha_7) \\
&= N (1-\gamma_2) \alpha_7 - \frac{\log(2N)}{\log 2} \\
&\geq 0
\end{align*}
and similarly
\begin{align*}
  2m - \gamma_3 n &= 2\nu_3(N!) - \gamma_3 \nu_2(N!) - \left(2\nu_3\left(\prod \tuple^{(2)}\right) - \gamma_3 \nu_2\left(\prod \tuple^{(2)}\right)\right) \\
&\geq 2\nu_3(N!) - \gamma_3 \nu_2(N!) - N (1-\gamma_3) \sum_{i=1}^6 \alpha_i \\
&> N - 2 \frac{\log N}{\log 3} - 2 - \gamma_3 N - N (1-\gamma_3) (1 - \alpha_7) \\ 
&= N (1-\gamma_3) \alpha_7 - 2\frac{\log(3N)}{\log 3} \\
&\geq 0.
\end{align*}
From \eqref{nm} and Cramer's rule we conclude that that $(n,2m)$ lies in the non-negative linear span of $(n_2, 2m_2)$, $(n_3, 2m_3)$, thus
\begin{equation}\label{comb}  (n,2m) = \beta_2 (n_2,2m_2) + \beta_3 (n_3,2m_3)\end{equation}
for some reals $\beta_2,\beta_3 \geq 0$.
We now create the multiset $\tuple^{(3)}$ by adding $\lfloor \beta_2 \rfloor$ copies of $2^{n_2} 3^{m_2}$ and $\lfloor \beta_3 \rfloor$ copies of $2^{n_3} 3^{m_3}$ to $\tuple^{(2)}$.  By \eqref{tlip}, this multiset remains $t$-admissible, and each element added increases the $t$-excess by at most $\kappa_*$.  The number of such elements can be upper bounded using \eqref{comb}, \eqref{legendre} as
\begin{align*}
  \lfloor \beta_2 \rfloor + \lfloor \beta_3 \rfloor &\leq \beta_2 + \beta_3 \\
  &\leq \frac{1}{\log t} \left( \beta_2 (n_2 \log 2 + m_2 \log 3) + \beta_3 (n_3 \log 2 + m_3 \log 3) \right) \\
  &= \frac{1}{\log t} (n \log 2 + m \log 3) \\
  &\leq \frac{1}{\log t} ((\nu_2(N!)-NB_2) \log 2 + (\nu_3(N!)-NB_3) \log 3) \\
  &\leq \frac{1}{\log t} \left(N \log 2 + \frac{N}{2} \log 3 - NB_2 \log 2 - NB_3 \log 3\right) \\
  &= \frac{N \log 12}{2\log t} - \frac{N(B_2 \log 2 + B_3 \log 3)}{\log t}.
\end{align*}
By \eqref{excess-2}, \eqref{delta7-def}, we thus have
\begin{equation}\label{excess-3}
 \frac{1}{N} \excess_t(\tuple^{(3)}) \leq \sum_{i=1}^7 \delta_i.
\end{equation}
Meanwhile by construction we see that $\tuple^{(3)}$ is a subfactorization of $N!$ that is in balance at all non-tiny primes, with tiny prime surpluses bounded by
$$ \nu_2\left( \frac{N!}{\prod \tuple^{(3)}}\right) \leq n_2+n_3; \quad \nu_3\left( \frac{N!}{\prod \tuple^{(3)}}\right) \leq m_2+m_3.$$
and thus by \eqref{tlip}, \eqref{delta8-def}, we thus have
$$ \frac{1}{N} \sum_p \nu_p\left( \frac{N!}{\prod \tuple^{(3)}}\right) \log p \leq \frac{\log 2^{n_2} 3^{m_2} + \log 2^{n_3} 3^{m_3}}{N} \leq \delta_8$$
and thus by \eqref{excess-3}, \eqref{delta-cond} we have
$$ \excess_t(\tuple^{(3)}) + \sum_p \nu_p\left( \frac{N!}{\prod \tuple^{(3)}}\right) \log p \leq \log N! - N \log t.$$
Applying \Cref{t-descrip}, we conclude that $t(N) \geq t$ as claimed.

\section{Estimating terms}\label{construction-sec}

In order to use \Cref{repair} for a given choice of $N,t$, we need to find a $t$-admissible tuple $\tuple^{(0)}$ and parameters $K, \kappa_*, \gamma_2, \gamma_3$ obeying \eqref{conditions} as well as good upper bounds on the quantities $\delta_i$, $i=1,\dots,8$ and $\alpha_i$, $i=1,\dots,7$, that can either be evaluated asymptotically or numerically.  Many of the terms here will be straightforward to estimate; we discuss only the more difficult ones.

We introduce a further natural number parameter $A$ and define 
\begin{equation}\label{sigma-def}
  \sigma \coloneqq \frac{3N}{At}.
\end{equation}
We let $\tuple^{(0)}$ be the multiset of $3$-rough elements of the interval $(t, t(1+\sigma)]$, with each element repeated precisely $A$ times.  This is clearly $t$-admissible.  It has no presence at tiny primes, so
\begin{equation}\label{alpha1-vanish}
  \alpha_1 = 0.
\end{equation}

We use the notation $\sum^*$ to denote summation restricted to $3$-rough numbers, thus for instance $\sum_{a < k \leq b}^* 1$ denotes the number of $3$-rough numbers in $(a,b]$.  We have a simple estimate for such counts:

\begin{lemma}\label{lit}  For any interval $(a,b]$ with $0 \leq a \leq b$ one has $\sum_{a < k \leq b}^* 1 = \frac{b-a}{3} + O_{\leq}(4/3)$.
\end{lemma}

\begin{figure}
  \centering
  \includegraphics[width=0.8\textwidth]{sawtooth.png}
  \caption{The function $\sum_{k \leq x}^* 1 - \frac{x}{3}$.}\label{fig-saw}
\end{figure}

\begin{proof}  By the triangle inequality, it suffices to show that $\sum_{0 < k \leq x}^* 1 - \frac{x}{3} = O_{\leq}(2/3)$ for all $x \geq 0$.  The claim is easily verified for $0 \leq x \leq 6$, and the left-hand side is $6$-periodic in $x$, giving the claim; see \Cref{fig-saw}.
\end{proof}

This lets us estimate $\delta_1$:

\begin{lemma}\label{delta1-bound} We have
$$ \delta_1 \leq \frac{3N}{2tA} + \frac{4}{N}.$$
\end{lemma}

\begin{proof}  By definition, we have
$$ \excess_t(\tuple^{(1)}) = A \sum^*_{t < n \leq t(1+\sigma)} \log \frac{n}{t}.$$
By the fundamental theorem of calculus, this is
$$ A \int_0^{t\sigma} \sum^*_{t < n \leq t+h} 1\ \frac{dh}{t+h}.$$
Bounding $\frac{1}{t+h}$ by $\frac{1}{t}$ and applying \Cref{lit}, \eqref{sigma-def}, we conclude that
$$
 \excess_t(\tuple^{(1)}) \leq A \int_0^{3N/A} \left(\frac{h}{3} + \frac{4}{3}\right) \frac{dh}{t} = \frac{3N^2}{2tA} + 4.
$$
and the claim follows.
\end{proof}

To construct $\gamma_2, \gamma_3, \kappa_*, n_2, m_2, n_3, m_3$, we introduce another parameter $L \geq 1$ and assume that
\begin{equation}\label{t-lower}
  t > 3L.
\end{equation}
We define $n_2,n_3,m_2,n_3$ by setting
$$
2^{n_2} 3^{m_2} \coloneqq 2^{n_0} \lceil t/2^{n_0} \rceil^{\langle 2,3 \rangle}; \quad 
2^{n_3} 3^{m_3} \coloneqq 3^{m_0} \lceil t/3^{m_0} \rceil^{\langle 2,3 \rangle}$$
where $2^{n_0}, 3^{m_0}$ are the largest powers of $2,3$ respectively that are at most $t/L$.  By construction and \eqref{kappa-def}, \eqref{tlip} holds with
\begin{equation}\label{kappas-eq}
  \kappa_* = \kappa_L.
\end{equation}
We have
$$  2m_2 \leq \frac{2}{\log 3} \log \lceil t/2^{n_0} \rceil^{\langle 2,3 \rangle} \leq \frac{2}{\log 3} (\log(2L) + \kappa_L) 
$$
and
$$n_2 \geq n_0 \geq \frac{\log t - \log(2L)}{\log 2};$$
similarly
$$ n_3 \leq \frac{1}{\log 2} (\log(3L)+\kappa_L)$$
and
$$ 2m_3 \geq \frac{2(\log t - \log(3L))}{\log 3}.$$
We conclude that \eqref{nm} holds with
\begin{equation}\label{gammas-def}
\begin{split}
\gamma_2 &\coloneqq \frac{2 \log 2}{\log 3} \frac{\log(2L) + \kappa_L}{\log t - \log(2L)} \\
\gamma_3 &\coloneqq \frac{\log 3}{2\log 2} \frac{\log(3L) + \kappa_L}{\log t - \log(3L)};
\end{split}
\end{equation}
one can of course also take larger values of $\gamma_2,\gamma_3$ if desired.
This lets us compute the quantity $\kappa_{**}$ defined in \eqref{kappastar-def}.

To estimate $\delta_3, \alpha_3$ we use

\begin{lemma}\label{val-bound} For every $3 < p \leq t/K$, one has
\begin{equation}\label{nup} 
  \nu_p\left(\frac{N!}{\prod \tuple^{(1)}}\right) =  
O_{\leq}\left(\frac{4A+3}{3} \left\lceil \frac{\log N}{\log p}  \right\rceil\right).
\end{equation}
\end{lemma}

\begin{proof}
One has
  \begin{align*}
    \nu_p(\prod \tuple^{(1)}) &= A \sum_{t < n \leq t(1+\sigma)}^* \nu_p(n) \\
    &= A \sum_{1 \leq j \leq \frac{\log N}{\log p}} \sum_{t/p^j < n \leq t(1+\sigma)/p^j}^* 1 \\
    &= A \sum_{1 \leq j \leq \frac{\log N}{\log p}} \left(\frac{N}{p^j A} + O_{\leq}(4/3)\right) \\
    &= \frac{N}{p-1} - O_{\leq}^+\left(\frac{1}{p-1}\right)
    + O_{\leq}\left(\frac{4A}{3} \left\lceil \frac{\log N}{\log p}  \right\rceil\right) \\
    &= \frac{N}{p-1} 
    - O_{\leq}^+\left(\left\lceil \frac{\log N}{\log p}  \right\rceil\right)
    + O_{\leq}\left(\frac{4A}{3} \left\lceil \frac{\log N}{\log p}  \right\rceil\right).
  \end{align*}
  Meanwhile, from \eqref{legendre} one has
  $$ \nu_p(N!) = \frac{N}{p-1} - O_{\leq}^+\left(\left\lceil \frac{\log N}{\log p}  \right\rceil\right)$$
and the claim follows.  
\end{proof}

\begin{corollary}\label{delta3-alpha3-bound} One has
$$ \delta_3 \leq \frac{(4A+3)\kappa_{4.5}}{3N} \left(\pi(t/K) + \frac{\log N}{\log 5} \pi(\sqrt{N})\right)$$
and
$$ \alpha_3 \leq \frac{2(4A+3)}{3N\log 12} \left(\log \frac{t}{K} + \kappa_{**}\right) \left(\pi(t/K) + \frac{\log N}{\log 5} \pi(\sqrt{N})\right).$$
\end{corollary}

\begin{proof} This is immediate from \Cref{val-bound} and \eqref{alpha3-def}, \eqref{delta3-def} after noting that $\lfloor \frac{\log N}{\log p} \rfloor \leq 1 + \frac{\log N}{\log 5} 1_{p \leq \sqrt{N}}$ for $3 < p \leq t/K$.
\end{proof}

The main quantities left to estimate are the quantities $\delta_4, \delta_5, \alpha_4, \alpha_5$ that involve $A_{p_1}$.  By construction of $\tuple^{(0)}$, we have
$$
A_{p_1} = \frac{1}{N} \sum_m^* \nu_{p_1}(m) \sum_{\frac{t}{K}, \frac{t}{m} < p \leq \frac{t(1+\sigma)}{m}} A.
$$
In particular, for $p > K(1+\sigma)$ the quantity $A_{p_1}$ vanishes entirely:
\begin{equation}\label{ap1-vanish}
  A_{p_1} = 0.
\end{equation}
For the remaining primes $3 < p \leq K(1+\sigma)$ one has
\begin{equation}\label{ap1-small}
A_{p_1} = \frac{A}{N} \sum_{m \leq K(1+\sigma)}^* \nu_{p_1}(m) \left( \pi\left(\frac{t(1+\sigma)}{m}\right) - \pi\left(\frac{t}{\min(m,K)} \right) \right).
\end{equation}
In practice, these expressions can be adequately controlled by \Cref{osc-lemma}, as can the quantities $B_{p_1}$.

\section{The asymptotic regime}

With the above estimates, we can now establish the lower bound in \Cref{main}(iv).  Thus we aim to show that $t(N) \geq t$ for sufficiently large $N$, where
\begin{equation}\label{main-lower}
   t \coloneqq \frac{N}{e} - \frac{c_0 N}{\log N} + \frac{N}{\log^{1+c_1} N} \asymp N
\end{equation}
and $0 < c_1 < 1$ is a small absolute constant. We use the construction of the previous section with the parameters
\begin{align}
  A &\coloneqq \lfloor \log^2 N \rfloor\label{a-asym}\\
K &\coloneqq \lfloor \log^3 N \rfloor \label{k-asym} \\
L &\coloneqq N^{0.1},\label{l-asym}
\end{align}
so from \eqref{sigma-def} one has
\begin{equation}\label{sigma-alt}
   \sigma = \frac{3N}{tA} \asymp \frac{1}{A} \asymp \frac{1}{\log^2 N}.
\end{equation}
The conditions \eqref{conditions}, \eqref{t-lower} are easily verified for $N$ large enough.

By \eqref{kappas-eq}, \eqref{l-asym}, and \Cref{power-lemma}(ii) we have
$$ \kappa_* \ll \log^{-c} N$$
for some absolute constant $c>0$.  From \eqref{gammas-def}, \eqref{main-lower}, \eqref{l-asym} we have
$$ \gamma_2 = \frac{1}{10} \frac{2\log 2}{\log 3} + O\left(\frac{1}{\log N}\right), \quad
 \gamma_3 = \frac{1}{10} \frac{\log 3}{2\log 2} + O\left(\frac{1}{\log N}\right)
$$
and hence by \eqref{kappastar-def},  \eqref{kappastar-2-def}, \eqref{kappastar-3-def}
$$ \kappa_{**} \ll 1.$$
By \Cref{repair}, it thus suffices to establish the inequalities \eqref{delta-cond}, \eqref{alpha-cond}.
Several of the quantities $\delta, \delta_i, \alpha_i$ can now be immediately estimated using \eqref{alpha1-vanish}, \eqref{delta1-bound}, \Cref{delta3-alpha3-bound}, \eqref{stirling}, and the prime number theorem:
\begin{align*}
  \delta_1 &\ll \frac{1}{A} \asymp \frac{1}{\log^2 N} \\
  \delta_3 &\ll \frac{A}{K \log N} \asymp \frac{1}{\log^2 N} \\
  \delta_6 &\ll \frac{1}{N} \\
  \delta_7 &\ll \frac{\kappa_*}{\log N} \ll \frac{1}{\log^{1+c} N} \\
  \delta_8 &\ll \frac{\log N}{N} \\
  \delta &= \frac{ec_0}{\log N} + \frac{e}{\log^{1+c_1} N} + O\left( \frac{1}{\log^2 N} \right)\\  
\end{align*}
\begin{align*}
  \alpha_1 &= 0 \\
  \alpha_3 &\ll \frac{A}{K} \asymp \frac{1}{\log N} \\
  \alpha_6 &\ll \frac{\log N}{N} \\
  \alpha_7 &\ll \frac{\log N}{N}.
\end{align*}
On the interval $(t/NK,1]$, the function $f_{N/t}$ is piecewise monotone with $O(K)$ pieces, and bounded by $1$, so its augmented total variation norm is $O(K)$.  Applying \eqref{delta2-def} and \Cref{osc-lemma} (with classical error term), we have
\begin{align*}
\delta_2 &\leq \frac{1}{\log(t/K)} \int_{t/NK}^1 f_{N/t}(x)\ dx + O\left( \frac{1}{\log^2 N} \right) \\
&\leq \frac{1}{\log N} \int_{1/eK}^{N/et} f_{N/t}(etx/N)\ dx + O\left( \frac{1}{\log^2 N} \right)
\end{align*}
where we have used \eqref{falpha-bound} to manage error terms.
As in the proof of \Cref{upper-bound}, the function $f_{N/t}(etx/N)$ differs from $f_e(x)$ outside of an exceptional set of measure $O(1/\log N)$, and hence by \eqref{c0-def} (and \eqref{falpha-bound}) we have
$$ \delta_2 \leq \frac{ec_0}{\log N} + O\left( \frac{1}{\log^2 N} \right).$$
To finish the verification of the conditions \eqref{delta-cond}, \eqref{alpha-cond}, it will suffice to show that
\begin{equation}\label{delta-remaining}
\delta_4, \delta_5 \ll \frac{(\log\log N)^{O(1)}}{\log^2 N}
\end{equation}
and
\begin{equation}\label{alpha-remaining}
\alpha_2, \alpha_4, \alpha_5 \ll \frac{(\log\log N)^{O(1)}}{\log N}.
\end{equation}
By Mertens' theorem (or \Cref{osc-lemma}) and \eqref{delta4-def}, \eqref{delta5-def}, \eqref{alpha2-def}, \eqref{alpha4-def}, \eqref{alpha5-def}, \eqref{sigma-alt}, it suffices to show that
\begin{equation}\label{ap1-bound}
A_{p_1}, B_{p_1} \ll \frac{(\log\log N)^{O(1)}}{p_1 \log N} 
\end{equation}
for all $p_1 \leq K(1+\sigma)$ (recalling from \eqref{ap1-vanish} that $A_{p_1}$ vanishes for any larger $p_1$), as well as the variant
\begin{equation}\label{ap1-diff}
  |A_{p_1}-B_{p_1}|_{0,1} \ll \frac{(\log\log N)^{O(1)}}{p_1 \log^2 N}
\end{equation}
for $3 < p_1 \leq K$.

For \eqref{ap1-bound} we use \eqref{ap1-small}, \eqref{B-def}, and the crude bound 
\begin{equation}\label{crude}
  \nu_{p_1}(m) \ll 1_{p_1|m} \log\log N
\end{equation}
 for $m \leq K(1+\sigma)$, and reduce to showing that
$$ \frac{A}{N} \sum_{m \leq K(1+\sigma)} 1_{p_1|m} \left( \pi\left(\frac{t(1+\sigma)}{m}\right) - \pi\left(\frac{t}{\min(m,K)} \right) \right).
\ll \frac{(\log\log N)^{O(1)}}{p_1 \log N}$$
and
$$ \frac{1}{N} \sum_{m \leq K} 1_{p_1|m} \sum_{\frac{t}{m} \leq p < \frac{t}{m-1}} \left \lfloor \frac{N}{p} \right\rfloor \ll
\frac{(\log\log N)^{O(1)}}{p_1 \log N}.$$ 
But from the Brun--Titchmarsh inequality (or \Cref{osc-lemma}) and \eqref{sigma-alt} one has
$$ \pi\left(\frac{t(1+\sigma)}{m}\right) - \pi\left(\frac{t}{\min(m,K)}\right) \ll \frac{t\sigma}{m \log N} \ll \frac{N}{Am \log N}$$
and
$$ \sum_{\frac{t}{m} \leq p < \frac{t}{m-1}} \left \lfloor \frac{N}{p} \right\rfloor \ll \frac{tm}{m^2 \log N} \ll \frac{N}{m \log N}$$
and the claim then follows from summing the harmonic series.

It remains to show \eqref{ap1-diff}.  For $3 < p_1 \leq K$, we see from \eqref{ap1-small}, \eqref{sigma-alt}, \eqref{crude} and \Cref{osc-lemma} (with classical error term) that
\begin{align*}
A_{p_1} &\geq \frac{1}{N} \sum_{m \leq K(1+\sigma)}^* \nu_{p_1}(m) \left( \frac{A t \sigma}{m \log N} + O\left( \frac{(\log\log N)^{O(1)}A t \sigma}{m \log^2 N} \right) \right) \\
&= \frac{1}{\log N} \sum_{m \leq K(1+\sigma)}^* \nu_{p_1}(m) \frac{3}{m}+ O\left( \frac{(\log\log N)^{O(1)}}{\log^2 N} \right) \\
&= \frac{1}{\log N} \sum_{m \leq K}^* \nu_{p_1}(m) \frac{3}{m}+ O\left( \frac{(\log\log N)^{O(1)}}{\log^2 N} \right) 
\end{align*}
and similarly from \eqref{B-def}, \eqref{crude}, and \Cref{osc-lemma} (again with classical error term) 
\begin{align*}
  B_{p_1} &\leq
  \frac{1}{N} \sum_{m \leq K} \nu_{p_1}(m) \sum_{\frac{t}{m} \leq p < \frac{t}{m-1}} \frac{N}{p}  \\
  &\leq \frac{1}{N} \sum_{m \leq K} \nu_{p_1}(m) \left( \frac{N}{\log(t/m)} \int_{t/m}^{t/(m-1)} \frac{dx}{x} + O\left( \frac{N}{\log^{10} N} \right) \right) \\
  &\leq \frac{1}{\log N} \sum_{m \leq K} \nu_{p_1}(m) \log \frac{m}{m-1} + O\left( \frac{(\log\log N)^{O(1)}}{\log^2 N} \right)
\end{align*}
so it will suffice to establish the inequality
\begin{equation}\label{key-ineq} 
   \sum_{m \leq K} \nu_{p_1}(m) \log \frac{m}{m-1} \leq \sum_{m \leq K}^* \nu_{p_1}(m) \frac{3}{m}
\end{equation}
for all $p_1 > 3$.
  
Writing $\nu_{p_1}(m) = \sum_{j \geq 1} 1_{p_1^j|m}$, it suffices to show that
  $$ \sum_{m \leq K; p^j|m} \frac{3}{m} 1_{(m,6)=1} - \log \frac{m}{m-1} \geq 0.$$
Making the change of variables $m = p_1^j n$, it suffices to show that
  $$ \sum_{n \leq K'} \frac{3}{n} 1_{(n,6)=1} - p_1^j \log \frac{p_1^j n}{p_1^j n - 1} \geq 0$$
  for any $K' > 0$.   Using the bound
  $$ \log \frac{p_1^jn}{p_1^jn - 1} = \int_{p_1^jn-1}^{p_1^jn} \frac{dx}{x} \leq \frac{1}{p_1^jn-1}$$
  and $p^j \geq 5$, we have
  $$ p_1^j \log \frac{p_1^j n}{p_1^j n - 1} \leq \frac{1}{n-0.2}$$
  and so it suffices to show that
  \begin{equation}\label{kb}
  \sum_{n \leq K'}^* \frac{3}{n} 1_{(n,6)=1} - \frac{1}{n-0.2} \geq 0.
  \end{equation}
  Since 
  $$ \sum_{n=1}^\infty \frac{1}{n-0.2}-\frac{1}{n} = \psi(0.8)-\psi(1) = 0.353473\dots,$$
  where $\psi$ here denotes the digamma function rather than the von Mangoldt summatory function, it will suffice to show that
  \begin{equation}\label{kb-2}
  \sum_{n \leq K'} \frac{3}{n} 1_{(n,6)=1} - \frac{1}{n} \geq 0.4.\end{equation}
  This can be numerically verified for $K' \leq 100$, with substantial room to spare for $K'$ large; see \Cref{fig:kb}. On a block $6a-1 \leq n \leq 6a+4$ with $a>1$, the sum is positive:
  \begin{align*}
  \sum_{6a-1 \leq n \leq 6a+4}^* \frac{3}{n}  - \frac{1}{n } &= \left(\frac{1}{6a-1} - \frac{1}{6a}\right) + \left(\frac{1}{6a-1} - \frac{1}{6a+2}\right)\\
  &\quad + \left(\frac{1}{6a+1} - \frac{1}{6a+3}\right) + \left(\frac{1}{6a+1} - \frac{1}{6a+4}\right) \\
  &\quad > 0.
  \end{align*}
The inequality for $K'>100$ is then easily verified from the $K' \leq 100$ data and the triangle inequality.
  
  \begin{figure}
  \centering
  \includegraphics[width=0.5\textwidth]{key_ineq.png}
  \caption{A plot of \eqref{kb}, \eqref{kb-2}.}
  \label{fig:kb}
  \end{figure}







\section{Guy--Selfridge conjecture}

We now establish the Guy--Selfridge conjecture $t(N) \geq N/3$ in the range
$$ N \geq N_0 \coloneqq 10^{11}.$$
We will apply \Cref{repair} with the construction in \Cref{construction-sec} and the choice of parameters
\begin{align*}
  t &\coloneqq N/3\\
  A &\coloneqq 190\\
  K &\coloneqq 252 \\
  L &\coloneqq 4.5;
\end{align*}
the choice of $A$ and $K$ was obtained after some numerical experimentation.  In particular, by \eqref{sigma-def} we have
$$ \sigma = \frac{3N}{At}=\frac{9}{190} =0.047368\dots$$
and so
$$ K(1+\sigma)=263.936\dots$$
One can readily check the required conditions \eqref{conditions}, \eqref{t-lower} for $N \geq N_0$, so it remains to verify the hypotheses \eqref{delta-cond}, \eqref{alpha-cond} of \Cref{repair} in this range.  Some of the quantities in these hypotheses involve sums over large ranges, such as $(t/K,N]$; but one can use \Cref{osc-lemma} to obtain adequate upper or lower bounds on such quantities, leaving one with sums over short ranges such as $p \leq K$ or $p \leq K(1+\sigma)$.  Our bounds are designed to be monotone non-increasing in $N$, so the task reduces to evaluating these short sums for $N=N_0$, which can be accomplished with simple computer code\footnote{\url{https://github.com/teorth/erdos-guy-selfridge/blob/main/src/python/calculations.py}}.   As it turns out, we will have a tiny bit of room to spare, in that we can show 
\begin{equation}\label{delta-final}
  \sum_{i=1}^8 \delta_i \leq 0.9729 \delta
\end{equation}
and 
\begin{equation}\label{alpha-final}
  \sum_{i=1}^7 \alpha_i \leq 0.9697.
\end{equation}
We now bound some of the terms appearing in the above expression. From \Cref{power-lemma} we have
$$ \kappa_{4.5} = \log \frac{4}{3} =0.28768\dots.$$
From \eqref{gammas-def} one can take
$$ \gamma_2 \coloneqq \frac{2 \log 2}{\log 3} \frac{\log(2L) + \kappa_L}{\log(N_0/3) - \log(2L)} = 0.1423165\dots$$
and
$$\gamma_3 \coloneqq \frac{\log 3}{2\log 2} \frac{\log(3L) + \kappa_L}{\log(N_0/3) - \log(3L)} =  0.1059116 \dots$$
and so by \eqref{kappastar-def} and some calculation
$$ \kappa_{**} \leq 6.830101\dots.$$
From \eqref{stirling} one has
$$ \delta \geq \log N - \log t = \log \frac{3}{e} = 0.0986122\dots.$$
From \eqref{delta1-bound} on has
$$ \delta_1 \leq \frac{9}{2A} + \frac{4}{N_0} \leq 0.240176 \delta.$$
From \eqref{delta2-def} and \Cref{osc-lemma} one has
\begin{align*}
  \delta_2 &\leq \frac{1}{\log(t/K)} \int_{1/3K}^1 f_3(x)\ dx + \frac{1}{\log(t/K)} \|f_3\|_{\mathrm{TV}((1/3K,1])} \frac{E(N)}{N} \\
  &\leq \frac{1}{\log(N_0/3K)} 0.919507 + \frac{1}{\log(N_0/3K)} 996.196 \frac{E(N_0)}{N_0} \\
  &\leq 0.50025 \delta.
\end{align*}
From \Cref{delta3-alpha3-bound} and \eqref{pi-upper} one has
\begin{align*}
\delta_3 &\leq \frac{(4A+3)\kappa_{4.5}}{3} \left(
\frac{1}{3K \log(N/3K)} + \frac{1.2762}{3K \log^2(N/3K)}
+ \frac{\log N}{\log 5 \sqrt{N} \log \sqrt{N}} + \frac{1.2762 \log N}{\log 5 \sqrt{N} \log^2 \sqrt{N}} \right) \\
&\leq \frac{(4A+3)\kappa_{4.5}}{3} \left(
  \frac{1}{3K \log(N_0/3K)} + \frac{1.2762}{3K \log^2(N_0/3K)}
  + \frac{\log N_0}{\log 5 \sqrt{N_0} \log \sqrt{N_0}} + \frac{1.2762 \log N_0}{\log 5 \sqrt{N_0} \log^2 \sqrt{N_0}} \right) \\
&\leq 0.059274 \delta.
\end{align*}
We skip $\delta_4, \delta_5, \delta_7$ for now.  From \eqref{delta6-def} we have
$$ \delta_6 \leq \frac{\kappa_{4.5}}{N_0} \leq 3 \times 10^{-11} \delta$$
and from \eqref{delta8-def} we have
$$ \delta_8 \leq \frac{2(\log(N_0/3) + \kappa_{4.5})}{N_0} \leq 6 \times 10^{-10} \delta$$
so these two terms are negligible in the analysis.

From \eqref{alpha1-vanish} we have
$$ \alpha_1 = 0.$$
We skip $\alpha_2, \alpha_4, \alpha_5$ for now.  From \Cref{delta3-alpha3-bound} and \eqref{pi-upper} one has
\begin{align*}
  \alpha_3 &\leq \frac{2(4A+3)}{3\log 12} \left(\log \frac{N}{3K} + \kappa_{**}\right)  \\
&\quad \times \left(
  \frac{1}{3K \log(N/3K)} + \frac{1.2762}{3K \log^2(N/3K)}
  + \frac{\log N}{\log 5 \sqrt{N} \log \sqrt{N}} + \frac{1.2762 \log N}{\log 5 \sqrt{N} \log^2 \sqrt{N}} \right).
\end{align*}
Expanding out the product, one can check that all terms are non-increasing in $N$; so we may substitute $N_0$ for $N$ in the right-hand side, which after some calculation gives
$$ \alpha_3 \leq 0.417501.$$
From \eqref{alpha6-def} we have
\begin{align*}
   \alpha_6 &\leq \frac{2}{\log 12} \left( \frac{\log(N_0/3)}{N_0} + \frac{\kappa_{**}}{N_0} \right) \\
   &\leq 3 \times 10^{-10}
\end{align*}
and similarly from \eqref{alpha7-def} we have
\begin{align*}
\alpha_7 &\leq \max\left( \frac{\log(2N_0)}{(1-\gamma_2)N_0\log 2},  \frac{2\log(3N_0)}{(1-\gamma_3)N_0\log 3}\right) \\
&\leq 6 \times 10^{-10}
\end{align*}
so the contribution of these two terms are negligible.

The remaining terms $\delta_4, \delta_5, \delta_7, \alpha_2, \alpha_4, \alpha_5$ to estimate involve the quantities $A_{p_1}, B_{p_1}$ defined in \eqref{A-def}, \eqref{B-def}.  For $B_{p_1}$, we can split it as
$$ B_{p_1} = \sum_{m \leq K} \nu_{p_1}(m) \sum_{k: a_{k,m} < b_{k,m}} k \frac{1}{N} (\pi( N b_{k,m} ) - \pi(N a_{k,m}))$$
where 
$$ a_{k,m} \coloneqq \max\left( \frac{1}{3m}-, \frac{1}{k} \right); \quad b_{k,m} \coloneqq \max\left( \frac{1}{3(m-1)}-, \frac{1}{k-1} \right)$$
where the $-$ denotes the subtraction of an infinitesimal quantity to reflect the restriction to the range $\frac{t}{m} \leq p < \frac{t}{m-1}$ rather than
$\frac{t}{m} < p \leq \frac{t}{m-1}$.  Using \Cref{osc-lemma}, we can upper bound this quantity by
$$ B_{p_1} \leq \sum_{m \leq K} \nu_{p_1}(m) \sum_{k: a_{k,m} < b_{k,m}} k \frac{1}{\log(N a_{k,m})} \left( a_{k,m}-b_{k,m} + 2\frac{E(N_0 b_{k,m})}{N_0} \right)$$
and lower bound it by
$$ B_{p_1} \geq \sum_{m \leq K} \nu_{p_1}(m) \sum_{k: a_{k,m} < b_{k,m}} k \frac{1}{\log(N b_{k,m})} \left( \left(1-\frac{2}{\sqrt{a_{k,m}}}\right) (a_{k,m}-b_{k,m}) + 2\frac{E(N_0 b_{k,m})}{N_0 b_{k,m}} \right).$$
After some calculation, these bounds inserted into \eqref{alpha2-def} give
$$ \alpha_2 \leq 0.260087$$
and in \eqref{delta7-def} they give
$$ \delta_7 \leq 0.112941 \delta.$$
As for the $A_{p_1}$, we know from \eqref{ap1-vanish} that this vanishes unless $3 < p_1 \leq K(1+\sigma)$.  From \eqref{ap1-small} and \Cref{osc-lemma} one has the upper bound
$$
A_{p_1} \leq \sum_{m \leq K(1+\sigma)}^* \frac{A \nu_{p_1}(m)}{\log(N/3\min(m,K))} \left( \frac{1+\sigma}{3m} - \frac{1}{3\min(m,K)} + \frac{2E(N_0(1+\sigma)/3m)}{N_0} \right) $$
and the lower bound
\begin{align*}
A_{p_1} &\geq \sum_{m \leq K(1+\sigma)}^* \frac{A \nu_{p_1}(m)}{\log(N(1+\sigma)/3m)} \times \\
& \quad \left( \left(1-\frac{2}{\sqrt{N_0(1+3\sigma)/3m}}\right) \left(\frac{1+\sigma}{3m} - \frac{1}{3\min(m,K)}\right) - \frac{2E(N_0(1+\sigma)/3m)}{N_0} \right).
\end{align*}
Using these bounds, one can verify by direct computation that $A_{p_1} \geq B_{p_1}$ for all $3 < p_1 \leq K(1+\sigma)$ (cf. \eqref{key-ineq}), and from \eqref{delta4-def}, \eqref{delta5-def}, \eqref{alpha4-def}, \eqref{alpha5-def} one can then verify that
\begin{align*}
\delta_4 &\leq 0.001212 \delta \\
\delta_5 &\leq 0.057447 \delta \\
\alpha_4 &\leq 0.008523 \\
\alpha_5 &\leq 0.283524
\end{align*}
and the claims \eqref{delta-final}, \eqref{alpha-final} follow by summing all the bounds.

  
\appendix

\section{Distance to the next \texorpdfstring{$3$}{3}-smooth number}\label{power-sec}

We now establish the various claims in \Cref{power-lemma}.  We begin with part (iii).  The claim \eqref{mod-kappa} is immediate from \eqref{kappa-def}, \eqref{fancy-kappa-def}.  Now prove \eqref{12-2}, \eqref{12-3}.  If we write $\lceil x/12^a \rceil^{\langle 2,3 \rangle} = 2^b 3^c$, then by \eqref{kappa-def} we have
$$ b \log 2 + c \log 3 \leq \log x - a \log 12 + \kappa_L,$$
while from definition of $a$ we have
\begin{equation}\label{xa12}
  \log x - a \log 12 \leq \log(12L).
\end{equation}
We now compute
\begin{align*}
  \frac{\nu_2(\lceil x \rceil^{\langle 2,3\rangle}_L) - 2 \gamma \nu_3(\lceil x \rceil^{\langle 2,3\rangle}_L)}{1-\gamma} 
  &= \frac{2a+b - 2\gamma(a+c)}{1-\gamma} \\
  &\leq 2a + \frac{\log x - a \log 12 + \kappa_L}{(1-\gamma) \log 2}  \\
  &= \frac{2\log x}{\log 12} + \left( \frac{1}{(1-\gamma)\log 2} - \frac{2}{\log 12}\right) \left(\log x - a \log 12\right)
  + \frac{\kappa_L}{(1-\gamma)\log 2} 
\end{align*}
giving \eqref{12-2} from \eqref{xa12}; similarly, we have
\begin{align*}
  \frac{2\nu_3(\lceil x \rceil^{\langle 2,3\rangle}_L) - \gamma \nu_2(\lceil x \rceil^{\langle 2,3\rangle}_L)}{1-\gamma} 
  &= \frac{2(a+c) - \gamma(2a+b)}{1-\gamma} \\
  &\leq 2a + \frac{2(\log x - a \log 12 + \kappa_L)}{(1-\gamma) \log 3}  \\
  &= \frac{2\log x}{\log 12} + \left( \frac{2}{(1-\gamma)\log 3} - \frac{2}{\log 12}\right) \left(\log x - a \log 12\right)
  + \frac{2\kappa_L}{(1-\gamma)\log 3} 
\end{align*}
giving \eqref{12-3} from \eqref{xa12}.

To prove parts (i) and (ii) of \Cref{power-lemma}, we establish the following lemma to upper bound $\kappa_L$.

\begin{lemma}\label{lemcount-0}  If $n_1,n_2,m_1,m_2$ are natural numbers such that $n_1+n_2, m_1+m_2 \geq 1$ and
$$ 1 \leq \frac{3^{m_1}}{2^{n_1}}, \frac{2^{n_2}}{3^{m_2}}$$
then
$$ \kappa_{\min( 2^{n_1+n_2},3^{m_1+m_2})/6} \leq \log \max\left(\frac{3^{m_1}}{2^{n_1}}, \frac{2^{n_2}}{3^{m_2}}\right).$$
\end{lemma}

\begin{proof}  If $\min( 2^{n_1+n_2},3^{m_1+m_2})/6 \leq t \leq 2^{n_2-1} 3^{m_1-1}$, then we have
\begin{equation}\label{tb} 
  t \leq 2^{n_2-1} 3^{m_1-1} \leq \max\left(\frac{3^{m_1}}{2^{n_1}}, \frac{2^{n_2}}{3^{m_2}}\right) t,
\end{equation}
so we are done in this case.  Now suppose that $t > 2^{n_2-1} 3^{m_1-1}$.
If we write $\lceil t \rceil^{\langle 2,3 \rangle} =2^n 3^m$ be the smallest $3$-smooth number that is at least $t$, then we must have $n \geq n_2$ or $m \geq m_1$ (or both).  Thus at least one of $\frac{2^{n_1}}{3^{m_1}} 2^n 3^m$ and $\frac{3^{m_2}}{3^{n_2}} 2^n 3^m$ is an integer, and is thus at most $t$ by construction.  This gives \eqref{tb}, and the claim follows.
\end{proof}

Some efficient choices of parameters for this lemma are given in \Cref{approx-table}.  For instance, $\kappa_{4.5} \leq \log\frac{4}{3} = 0.28768\dots$ and $\kappa_{40.5} \leq \log \frac{32}{27} = 0.16989\dots$.  In fact, since $\lceil 4.5+\eps \rceil^{\langle 2,3\rangle} = 6$ and $\lceil 40.5+\eps \rceil^{\langle 2,3\rangle} = 48$ for all sufficiently small $\eps>0$, we see that these bounds are sharp (And similarly for the other entries in \Cref{approx-table}); this establishes part (i).

\begin{table}[ht]
\centering
\begin{tabular}{|c|c|c|c|c|c|}
\hline
$n_1$ & $m_1$ & $n_2$ & $m_2$ & $\min(2^{n_1+n_2},3^{m_1+m_2})/6$ & $\log \max(3^{m_1}/2^{n_1}, 2^{n_2}/3^{m_2})$ \\
\hline
$1$ & $1$ & $\mathbf{1}$ & $\mathbf{0}$ & $1/2 = 0.5$ & $\log 2 = 0.69314\dots$ \\
\hline
$\mathbf{1}$ & $\mathbf{1}$ & $2$ & $1$ & $2^2/3 = 1.33\dots$ & $\log (3/2) = 0.40546\dots$\\
\hline
$3$ & $2$ & $\mathbf{2}$ & $\mathbf{1}$ & $3^2/2 = 4.5$ & $\log (2^2/3) = 0.28768\dots$ \\
$3$ & $2$ & $\mathbf{5}$ & $\mathbf{3}$ & $3^4/2 = 40.5$ & $\log (2^5/3^3) = 0.16989\dots$ \\
\hline
$\mathbf{3}$ & $\mathbf{2}$ & $8$ & $5$ & $2^{10}/3 = 341.33\dots$ & $\log (3^2/2^3) = 0.11778\dots$\\ 
$\mathbf{11}$ & $\mathbf{7}$ & $8$ & $5$ & $2^{18}/3 = 87381.33\dots$ & $\log (3^7/2^{11}) = 0.06566\dots$ \\
\hline
$19$ & $12$ & $\mathbf{8}$ & $\mathbf{5}$ & $3^{17}/2 \approx 6.4 \times 10^7$ & $\log (2^8/3^5) = 0.05211\dots$ \\
$19$ & $12$ & $\mathbf{27}$ & $\mathbf{17}$ & $3^{29}/2 \approx 3.4 \times 10^{13}$ & $\log (2^{27}/3^{17}) = 0.03856\dots$ \\
$19$ & $12$ & $\mathbf{46}$ & $\mathbf{29}$ & $3^{41}/2 \approx 1.8 \times 10^{19} $ & $\log (2^{46}/3^{29}) = 0.02501\dots$ \\
\hline
\end{tabular}
\caption{Efficient parameter choices for \Cref{lemcount-0}.  The parameters used to attain the minimum or maximum are indicated in \textbf{boldface}. Note how the number of rows in each group matches the terms $1,1,2,2,3,\dots$ in the continued fraction expansion.}\label{approx-table}
\end{table}

\begin{remark}
It should be unsurprising that the continued fraction convergents $1/1$, $2/1$, $3/2$, $8/5$, $19/12$, $\dots$ to 
$$\frac{\log 3}{\log 2} = 1.5849\dots = [1; 1,1,2,2,3,1,\dots]$$
are often excellent choices for $n_1/m_1$ or $n_2/m_2$, although other approximants such as $5/3$ or $11/7$ are also usable.
\end{remark}

Finally, we establish (ii).  From the classical theory of continued fractions, we can find rational approximants
\begin{equation}\label{abn}
 \frac{p_{2j}}{q_{2j}} \leq \frac{\log 3}{\log 2} \leq \frac{p_{2j+1}}{q_{2j+1}}
\end{equation}
to the irrational number $\log 3/\log 2$, where the convergents $p_j/q_j$ obey the recursions
$$ p_j = b_j p_{j-1} + p_{j-2}; \quad q_j = b_j q_{j-1} + q_{j-2}$$
with $p_{-1} = 1, q={-1}=0, p_0 = b_0, q_0=1$, and 
$$[b_0;b_1,b_2,\dots] = [1;1,1,2,2,3,1\dots]$$ 
is the continued fraction expansion of $\frac{\log 3}{\log 2}$.  Furthermore, $p_{2j+1}q_{2j} - p_{2j} q_{2j+1} = 1$, and hence
\begin{equation}\label{abn-2} 
  \frac{\log 3}{\log 2} - \frac{p_{2j}}{q_{2j}} = \frac{1}{q_{2j} q_{2j+1}}.
\end{equation}
By Baker's theorem, $\frac{\log 3}{\log 2}$ is a Diophantine number, giving a bound of the form
\begin{equation}\label{bj1}
   q_{2j+1} \ll q_{2j}^{O(1)}
\end{equation}
and a similar argument (using $p_{2j+2} q_{2j+1}-p_{2j+1} q_{2j+2} = -1$) gives
\begin{equation}\label{bj2}
 q_{2j+2} \ll q_{2j+1}^{O(1)}.
\end{equation}
We can rewrite \eqref{abn} as
$$ 1 \leq \frac{3^{q_{2j}}}{2^{p_{2j}}}, \frac{2^{p_{2j+1}}}{3^{q_{2j+1}}}$$
and routine Taylor expansion using \eqref{abn-2} gives the upper bounds
$$ \frac{3^{q_{2j}}}{2^{p_{2j}}}, \frac{2^{p_{2j+1}}}{3^{q_{2j+1}}}\leq \exp\left( O\left( \frac{1}{q_{2j}}\right)\right).$$
From \Cref{lemcount-0} we obtain
$$
\kappa_{\min(2^{p_{2j} + p_{2j+1}}, 3^{q_{2j}+q_{2j+1}})/6} \ll \frac{1}{q_{2j}}.$$
The claim then follows from \eqref{bj1}, \eqref{bj2} (and the obvious fact that $\kappa$ is monotone non-increasing after optimizing in $j$.

\begin{remark}
It seems reasonable to conjecture that $c$ can be taken to be arbitrarily close to $1$, but this is essentially equivalent to the open problem of determining that irrationality measure of $\log 3 / \log 2$ is equal to $2$.
\end{remark}



\section{Estimating sums over primes}\label{primes-sec}

In this appendix we establish \Cref{osc-lemma}.  The key tool is

\begin{lemma}[Integration by parts]\label{integ-lemma}  Let $(y,x]$ be a half-open interval in $(0,+\infty)$.  Suppose that one has a function $a \colon \N \to \R$ and a continuous function $f: (y,x] \to \R$ such that 
  $$ \sum_{y < n \leq z} a_n = \int_z^y f(t)\ dt + C + O_{\leq}(A)$$
  for all $y \leq z \leq x$, and some $C \in \R$, $A>0$.  Then, for any function $b: (y,x] \to \R$ of bounded total variation, one has
\begin{equation}\label{tve}
   \sum_{y < n \leq x} b(n) a_n = \int_x^y b(t) f(t)\ dt + O_{\leq}(A\|b\|_{\mathrm{TV}^*(y,x]}).
\end{equation}
\end{lemma}

\begin{proof}  If, for every natural number $y < n \leq x$, one modifies $b$ to be equal to the constant $b(n)$ in a small neighborhood of $n$, then one does not affect the left-hand side of \eqref{tve} or increase the total variation of $b$, while only modifying the integral in \eqref{tve} by an arbitrarily small amount.  Hence, by the usual limiting argument, we may assume without loss of generality that $b$ is locally constant at each such $n$.  If we define the function $g \colon (y,x] \to \R$ by
$$ g(z) \coloneqq  \sum_{y < n \leq z} a_n - \int_z^y f(u)\ du - C$$
then $g$ has jump discontinuities at the natural numbers, but is otherwise continuously differentiable, and is also bounded uniformly in magnitude by $A$.  We can then compute the Riemann--Stieltjes integral
$$ \int_{(y,x]} b\ dg = \sum_{y < n \leq x} b(n) a_n - \int_y^x f(t) b(t)\ dt.$$
Since the discontinuities of $g$ and $b$ do not coincide, we may integrate by parts to obtain
$$ \int_{(y,x]} b\ dg = b(x) g(x) - b(y^+) g(y^+) - \int_{(y,x]} g db.$$
The left-hand side is $O_{\leq}(A \|b\|_{\mathrm{TV}^*(y,x]})$, and the claim follows.
\end{proof}

We now prove \eqref{bv-exact}. In fact we prove the sharper estimate
\begin{equation}\label{bsharp}  
  \sum_{y < p \leq x} b(p) \log p = \int_y^x b(t) (1 - \frac{2}{\sqrt{t}}) \ dt
+ O_\leq\left(\|b\|_{\mathrm{TV}^*((y,x])} \tilde E(x) \right)
\end{equation}
where
\begin{equation}\label{etil-def}
\tilde E(x) \coloneqq 0.95 \sqrt{x} + \min( \max(\eps_0,\eps_1(x)), \eps_2(x), \eps_3(x)) 1_{x \geq 10^{19}}
\end{equation}
and
\begin{align*}
  \eps_0(x) &\coloneqq \frac{\sqrt{x}}{8\pi} \log x(\log x - 3)\\
  \eps_1(x) &\coloneqq 1.12494 \times 10^{-10}\\
  \eps_2(x) &\coloneqq 9.39 (\log^{1.515} x) \exp(-0.8274\sqrt{\log x})\\
  \eps_3(x) &\coloneqq 0.026 (\log^{1.801} x) \exp(-0.1853 (\log^{3/5} x) (\log\log x)^{-1/5})
\end{align*}
From using the $\eps_2$ term, it is clear that
$$ \tilde E(x) \ll x \exp(-c \sqrt{\log x})$$
for some absolute constant $c>0$; and by using the $\eps_0, \eps_1$ term and routine calculations one can show that 
$$ \tilde E(x) \leq E(x)$$
for all $x \geq 1423$.

Observe that $\tilde E$ is monotone non-decreasing. Thus by \Cref{integ-lemma}, to show \eqref{bsharp} will suffice to show that
$$ \sum_{p \leq x} \log p = x - \sqrt{x} + O_{\leq}(\tilde E(x))
  = \int_0^x \left(1-\frac{2}{\sqrt{t}}\right)\ dt + O_{\leq}(\tilde E(x))$$
for all $x \geq 1423$.

For $1423 \leq x \leq 10^{19}$, this claim follows from \cite[Theorem 2]{buthe-2}.  For $x > 10^{19}$, we apply \cite[(6.10), (6.11)]{buthe} to conclude that
$$
\sum_{p \leq x} \log p = \psi(x) - \psi(\sqrt{x}) + O_{\leq}(1.03883 (x^{1/3} + x^{1/5} + 2 (\log x) x^{1/13})),$$
where $\psi(x) \coloneqq \sum_{n \leq x} 
\Lambda(n)$ is the usual von Mangoldt summatory function.  
From \cite[Theorems 10,12]{rs} we have
$$ \psi(\sqrt{x}) = \sqrt{x} + O_{\leq}(0.18 \sqrt{x}).$$
Since
$$ 0.18 \sqrt{x} + 1.03883 (x^{1/3} + x^{1/5} + 2 (\log x) x^{1/13}) \leq 0.95 \sqrt{x}$$
in this range of $x$, it suffices to show that
$$ \psi(x) = x + O_{\leq}(\min( \max(\eps_0(x),\eps_1(x)), \eps_2(x), \eps_3(x)) )$$
for $x > 10^{19}$.  The claims for $i=2,3$ follow from \cite[Theorems 1.1, 1.4]{johnston-yang}.  In \cite[Theorem 2, (7.3)]{buthe}, the bound
$$ \psi(x) = x + O_{\leq}(\eps_0(x))$$
is established whenever $x \geq 5000$ and $4.92 \frac{x}{\sqrt{\log x}} \leq T$, where $T$ is a height up to which the Riemann hypothesis has been established.  Using the value $T = 3 \times 10^{12}$ from \cite{platt-rh}, we can therefore cover the range $10^{19} < x < e^{55}$ (in fact we could go up to $e^{58.33} \approx 2.1 \times 10^{25}$).  For $x \geq e^{55}$, we can use \cite[Table 2]{buthe} (the value $T = 2.445 \times 10^{12}$ used there following from \cite{platt-rh}).

\begin{remark} Assuming the Riemann hypothesis, the $\eps_1, \eps_2, \eps_3$ terms in the definition of $\tilde E(x)$ may be deleted, since \cite[(7.3)]{buthe} then holds for all $x \geq 5000$.  
\end{remark}

The claim \eqref{pix} now follows from \eqref{bv-exact} by setting $b(t) \coloneqq \frac{1}{\log t}$.  For non-negative $b$, the claims \eqref{bv-upper}, \eqref{bv-lower} follow from \eqref{bv-exact} and the pointwise bounds
$$ \frac{1}{\log x} b(p) \log p \leq b(p) \leq \frac{1}{\log y} b(p) \log p$$
and
$$ 1-\frac{2}{\sqrt{y}} \leq 1-\frac{2}{\sqrt{t}} \leq 1.$$
Finally, \eqref{pixy-upper}, \eqref{pixy-lower} come from specializing \eqref{bv-upper}, \eqref{bv-lower} to the case of an indicator function $b = 1_{(y,x]}$.



\section{Computation of \texorpdfstring{$c_0$}{c\_0}}\label{c0-app}

In this appendix we give some details regarding the numerical estimation of the constant $c_0$ defined in \eqref{c0-def}.  As one might imagine from an inspection of \Cref{fig-mean}, direct application of numerical quadrature converges quite slowly due to the oscillatory singularity.  To resolve the singularity, we can perform a change of variables $x=1/y$ to express $c_0$ as an improper integral:
\begin{equation}\label{c0-alt}
   c_0 = \frac{1}{e} \int_1^\infty \lfloor y \rfloor \log \frac{\lceil y/e \rceil}{y/e}\ \frac{dy}{y^2}.
\end{equation}
Next, observe\footnote{We thank an anonymous commentor on the blog of one of the authors for this suggestion.} that
\begin{align*}
  \frac{1}{e} \int_e^\infty y \log \frac{\lceil y/e \rceil}{y/e}\ \frac{dy}{y^2}
  &= \sum_{k=1}^\infty \int_{ke}^{(k+1)e} y \log \frac{k+1}{y/e}\ \frac{dy}{y^2} \\
  &= \frac{1}{e} \sum_{k=1}^\infty \int_k^{k+1} \frac{\log(k+1)-\log y}{y}\ dy\\
  &= \frac{1}{2e} \sum_{k=1}^\infty \log^2 \frac{k+1}{k}\\
  &= 0.1797439053\dots;
\end{align*}
the convergence of this series can be accelerated by any number of methods, for instance by subtracting the asymptotically similar sum $\sum_{k=1}^\infty \frac{1}{k^2} = \frac{\pi^2}{6}$.  One also has
$$ \frac{1}{e} \int_1^e \lfloor y \rfloor \log \frac{e}{y}\ \frac{dy}{y^2} = \frac{2}{e^2} - \frac{\log 2}{2e} = 0.143173268\dots$$
and hence
$$ c_0 = \frac{1}{2e} \sum_{k=1}^\infty \log^2 \frac{k+1}{k}
+ \frac{2}{e^2} - \frac{\log 2}{2e} - \frac{1}{e} \int_e^\infty \{y\} \log \frac{\lceil y/e \rceil}{y/e}\ \frac{dy}{y^2}$$
where $\{x\} \coloneqq x - \lfloor x \rfloor$.  The integrand here decays like $O(1/y^3)$, and so one can numerically evaluate the integral with high confidence as
$$ \frac{1}{e} \int_e^\infty \{y\} \log \frac{\lceil y/e \rceil}{y/e}\ \frac{dy}{y^2} = 0.01849816\dots$$
so that
$$ c_0 = 0.30441894\dots.$$
Similar methods could be used to compute more digits of $c_1$, although the formulae are more complicated.

\begin{thebibliography}{10}

  
\bibitem{algr77}
K. Alladi, C. Grinstead, \emph{On the decomposition of $n!$ into prime powers}, J. Number Theory \textbf{9} (1977) 452--458.

\bibitem{bincover}
S. F. Assmann, D. S. Johnson, D. J. Kleitman, J. Y.-T. Leung, \emph{On a dual version of the one-dimensional bin packing problem}, J. Algorithms \textbf{5} (1984) 502--525.

\bibitem{buthe}
J. B\"uthe, \emph{Estimating $\pi(x)$ and related functions under partial RH assumptions}, Math. Comp., 85(301), 2483--2498, Jan. 2016.

\bibitem{buthe-2}
J. B\"uthe, \emph{An analytic method for bounding $\psi(x)$
}. Math. Comp., \textbf{87} (312), 1991--2009.


%\bibitem{dusart}
%P. Dusart, \emph{In\'egalit\'es explicites pour $\psi(x)$, $\theta(x)$, $\pi(x)$ et les nombres premier}, C. R. Math. Rep. Acad. Sci. Canada Vol. \textbf{21} (2) 1999, pp. 53--59.


\bibitem{dusart}
P. Dusart, \emph{Explicit estimates of some functions over primes}, Ramanujan J. \textbf{45} (2018) 227--251.

\bibitem{erdos-71}
P. Erd\H{o}s, \emph{Some problems in number theory}, in Computers in Number Theory, Academic Press, London New York, 1971, pp. 405--414.

\bibitem{erdos-96}
P. Erd\H{o}s, \emph{Some problems I presented or planned to present in my short talk}, Analytic number theory, Vol. 1 (Allerton Park, IL, 1995) (1996), 333--335.

\bibitem{erdos-graham}
P. Erd\H{o}s, R. Graham, \emph{Old and new problems and results in combinatorial number theory}, Monographies de L'Enseignement Mathematique 1980.

\bibitem{guy}
R. K. Guy, \emph{Unsolved Problems in Number Theory}, 3rd Edition, Springer, 2004.

\bibitem{guy-selfridge}
R. K. Guy, J. L. Selfridge, \emph{Factoring factorial $n$}, Amer. Math. Monthly \textbf{105} (1998) 766--767.

\bibitem{johnston-yang}
D. Johnston, A. Yang, \emph{Some explicit estimates for the error term in the prime number theorem}, J. Math. Anal. Appl., \textbf{527} (2) (2023), Paper No. 127460.

\bibitem{platt-rh}
D. Platt, T. Trudgian, \emph{The Riemann hypothesis is true up to $3\cdot10^{12}$}, Bull. Lond. Math. Soc. \textbf{53} (2021), no. 3, 792--797.

\bibitem{robbins}
H. Robbins, \emph{A Remark on Stirling's Formula}, Amer. Math. Monthly \textbf{62} (1955) 26--29.

%\bibitem{faber-kadiri}
%L. Faber, H. Kadiri, \emph{Corrigendum to: New bounds for $\psi(x)$}, Math. Comp. \textbf{87} (2018) 1451--1455.

\bibitem{rs}
J. Rosser, L. Schoenfeld, \emph{Approximate formulas for some functions of prime numbers}, Illinois J. Math. 6 (1962), 64--94.

\bibitem{tao}
T. Tao, \emph{Decomposing factorials into bounded factors}, preprint, 2025. \url{https://arxiv.org/abs/2503.20170v2}

\bibitem{github}
T. Tao, \emph{Verifying the Guy--Selfridge conjecture}, Github repository, 2025.  \url{https://github.com/teorth/erdos-guy-selfridge}.

\end{thebibliography}


\end{document}
